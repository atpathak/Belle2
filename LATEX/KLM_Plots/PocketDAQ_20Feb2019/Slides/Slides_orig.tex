%%%%%%%%%%%%%%%%%%%%%%%%%%%%%%%%%%%%%%%%%
% Beamer Presentation
% LaTeX Template
% Version 1.0 (10/11/12)
%
% This template has been downloaded from:
% http://www.LaTeXTemplates.com
%
% License:
% CC BY-NC-SA 3.0 (http://creativecommons.org/licenses/by-nc-sa/3.0/)
%
%%%%%%%%%%%%%%%%%%%%%%%%%%%%%%%%%%%%%%%%%

%----------------------------------------------------------------------------------------
%	PACKAGES AND THEMES
%----------------------------------------------------------------------------------------

\documentclass{beamer}

\mode<presentation> {

% The Beamer class comes with a number of default slide themes
% which change the colors and layouts of slides. Below this is a list
% of all the themes, uncomment each in turn to see what they look like.

\usetheme{default}
%\usetheme{AnnArbor}
%\usetheme{Antibes}
%\usetheme{Bergen}
%\usetheme{Berkeley}
%\usetheme{Berlin}
%\usetheme{Boadilla}
%\usetheme{CambridgeUS}
%\usetheme{Copenhagen}
%\usetheme{Darmstadt}
%\usetheme{Dresden}
%\usetheme{Frankfurt}
%\usetheme{Goettingen}
%\usetheme{Hannover}
%\usetheme{Ilmenau}
%\usetheme{JuanLesPins}
%\usetheme{Luebeck}
%\usetheme{Madrid}
%\usetheme{Malmoe}
%\usetheme{Marburg}
%\usetheme{Montpellier}
%\usetheme{PaloAlto}
%\usetheme{Pittsburgh}
%\usetheme{Rochester}
%\usetheme{Singapore}
%\usetheme{Szeged}
%\usetheme{Warsaw}

% As well as themes, the Beamer class has a number of color themes
% for any slide theme. Uncomment each of these in turn to see how it
% changes the colors of your current slide theme.

%\usecolortheme{albatross}
%\usecolortheme{beaver}
%\usecolortheme{beetle}
%\usecolortheme{crane}
%\usecolortheme{dolphin}
%\usecolortheme{dove}
%\usecolortheme{fly}
%\usecolortheme{lily}
%\usecolortheme{orchid}
%\usecolortheme{rose}
%\usecolortheme{seagull}
%\usecolortheme{seahorse}
%\usecolortheme{whale}
%\usecolortheme{wolverine}

%\setbeamertemplate{footline} % To remove the footer line in all slides uncomment this line
\setbeamertemplate{footline}[page number] % To replace the footer line in all slides with a simple slide count uncomment this line

\setbeamertemplate{navigation symbols}{} % To remove the navigation symbols from the bottom of all slides uncomment this line
}

%\usepackage[greek,german]{babel}
\usepackage{graphicx} % Allows including images
\usepackage{booktabs} % Allows the use of \toprule, \midrule and \bottomrule in tables
\usepackage{amsmath}
\usepackage{slashed}
\usepackage{color}
\usepackage{rotating}
%----------------------------------------------------------------------------------------
%	TITLE PAGE
%----------------------------------------------------------------------------------------

\title{Benchmarking reconstruction times vs. pileup} % The short title appears at the bottom of every slide, the full title is only on the title page

\author{\underline{Author} \and Author} % Your name

\author{{\bf A.~Pathak} and S.~Banerjee}


\institute[U. Louisville] % Your institution as it will appear on the bottom of every slide, may be shorthand to save space


\date{{Upgrade Tracking Meeting}\\\today\\} % Date, can be changed to a custom date

\pgfdeclareimage[height=.75cm]{biglogo}{/afs/cern.ch/user/s/swaban/public/university-of-louisville-logo}
\titlegraphic{\pgfuseimage{biglogo}} 

\begin{document}

\begin{frame}
\titlepage % Print the title page as the first slide
\end{frame}
%------------------------------------------------
\begin{frame}
\frametitle{Introduction}
\vspace*{0.0cm}
\begin{center}
\begin{itemize}
%\item This is concluding part of my authorship qualification task.
\item Reconstruction times should preferably be benchmarked in a 
 controlled environment and not in a shared cluster like LXPLUS. For 
 this study, the reconstruction jobs were run  as sole user in time
slots granted via dedicated run time requests.
\item The cluster is Intel Xeon 7210, 1.3GHz 64-core CPU, 
  116GB RAM, 200GB SSD disk, 5.5TB raid array (CentOS 7.3.1611).
\item The jobs use $\sim$ 4 to 5 GB RAM. Up to 20 jobs were run in
  parallel. Each job was configured to run on a single processor. 
\item 1000 $t\bar{t}$ events are studied for 4 pile-up in ITk and 1 in Run2.
%PBS-like batch system has been installed 
%this week to automatize this configuration. But for the results
%presented here, jobs were run interactively via nohup. Batch system 
%will make future study easier.
%\item Thanks to Dr.~Dang for installing ATLAS software using {\tt cvmfs} 
  %    running with squid cache in our new clusters at UofL.
\item {\bf HEP-SPEC06 score is 3.50 $\pm$ 0.15} (single core test),
      averaged over 5 random sets of cores. For 32 (64) core tests, scores are 3.64 (3.34)/core,
      consistent with single core tests.
\end{itemize}
\end{center}
\end{frame}
%------------------------------------------------
\begin{frame}
\frametitle{Introduction}
\vspace*{0.2cm}
\begin{itemize}
%\item This work is part of my authorship qualification task, which is
  %work in progress
%\item Reconstruction times should prefereably be benchmarked in
  %a controlled environment, eg. uniform cpuEfficiency/job
%\item Many thanks to Dr.~Dang for installing ATLAS software using {\tt
  %  cvmfs} in the new computing clusters at UofL.
%\item 1000 $t\bar{t}$ events are studied for 4 pile-up in ITk and 1 in Run2.
\item {\small RAWtoESD output of 20 jobs run over 50 events and 1 event are studied and then 
we subtract the total time used for 1 event from the time taken by 50 events.}% and divide 
%the resulting time by 49 to get time taken per event.} % from 7 sets of jobs. Each set 
 %consists of 30 jobs run over (1,2,5,10,20,50,100) events, respectively.
\item The following 4 pile-up configurations for ITk are studied:
\end{itemize}
%\vspace*{.5cm}
%\caption{Configuration}
%\label{tab:config}
\begin{center}
{\scalebox{1.03}{
\begin{tabular}{llll}\hline\hline
Pile-up & Reco release & Reco flag & Input RDO  \\\hline
$\langle \mu \rangle$ = 0    & 20.20.10.9    & AMI=r9877 & r-tag=9877 \\
$\langle \mu \rangle$ = 60 & 20.20.10.9   & AMI=r9876 & r-tag=9876 \\
$\langle \mu \rangle$ = 140         & 20.20.10.9    & AMI=r9875 & r-tag=9875 \\
$\langle \mu \rangle$ = 200         & 20.20.10.9    & AMI=r9871 & r-tag=9871 \\
\hline \hline 
\end{tabular}
}}
\vspace*{.1cm}
{\scriptsize{Samples: \url{mc15_14TeV.117050.PowhegPythia_P2011C_ttbar.recon.RDO.e2176_s3185_s3186_r9877/6/5/1}}}
\vspace*{.1cm}
\begin{itemize}
\item The following 1 pile-up configurations for Run2 is studied:
\end{itemize}
\vspace*{.1cm}
\begin{tabular}{llll}\hline\hline
Pile-up & Reco release & Reco flag & Input RDO  \\\hline
$\langle \mu \rangle$ = 20    & 20.7.5.1    & AMI=r7725 & r-tag=7725 \\
\hline \hline
\end{tabular}

\vspace*{.1cm}
{\scriptsize{Samples: \url{mc15_13TeV.410000.PowhegPythiaEvtGen_P2012_ttbar_hdamp172p5_nonallhad.recon.RDO.e3698_s2608_s2183_r7725}}}
\vspace*{.1cm}
\end{center}



%\vspace*{-.35cm}
%\begin{center}
%All configurations use for seeding:
%SiCombinatorialTrackFinderTool\_xk-01-00-34-06,
%SiSpacePointsSeedTool\_xk-01-00-35-05, 
%SiSPSeededTrackFinder-01-00-07
%\end{center}
\end{frame}
%------------------------------------------------
\begin{frame}
\begin{Large}
\centerline{Reconstruction times for ITk studies r9871}
\end{Large}
\end{frame}
%-------------------------------------------------
\begin{frame}
\frametitle{Times/event (seconds) for AthAlgSeq}
\begin{center}
\begin{normalsize}
Full Detector (left)  and Inner Detector(right) \\
\vspace*{-.2cm}
\begin{center}
\includegraphics[width=.40\textwidth,height=.40\textheight,type=png,ext=.png,read=.png]{/afs/cern.ch/user/a/atpathak/afswork/public/Pixel/ALLTXT_6Oct2017_r9871/pic/mydata_AthAlgSeq_49}
\includegraphics[width=.40\textwidth,height=.40\textheight,type=png,ext=.png,read=.png]{/afs/cern.ch/user/a/atpathak/afswork/public/Pixel/ALLTXT_6Oct2017_r9871_ID/pic/mydata_AthAlgSeq_49}

\end{center}
\end{normalsize}
\end{center}
\end{frame}
%-------------------------------------------------
\begin{frame}
\frametitle{Times/event (seconds) for Si Track Finder}
\begin{center}
\begin{normalsize}
Full Detector (left)  and Inner Detector (right) \\
\vspace*{-.2cm}
\begin{center}
\includegraphics[width=.40\textwidth,height=.40\textheight,type=png,ext=.png,read=.png]{/afs/cern.ch/user/a/atpathak/afswork/public/Pixel/ALLTXT_6Oct2017_r9871/pic/mydata_InDetSiSpTrackFinderSLHC_49}
\includegraphics[width=.40\textwidth,height=.40\textheight,type=png,ext=.png,read=.png]{/afs/cern.ch/user/a/atpathak/afswork/public/Pixel/ALLTXT_6Oct2017_r9871_ID/pic/mydata_InDetSiSpTrackFinderSLHC_49}

\end{center}
\end{normalsize}
\end{center}
\end{frame}
%-------------------------------------------------
\begin{frame}
\frametitle{Times/event (seconds) for Ambiguity Solver}
\begin{center}
\begin{normalsize}
Full Detector (left)  and Inner Detector (right) \\
\vspace*{-.2cm}
\begin{center}
\includegraphics[width=.40\textwidth,height=.40\textheight,type=png,ext=.png,read=.png]{/afs/cern.ch/user/a/atpathak/afswork/public/Pixel/ALLTXT_6Oct2017_r9871/pic/mydata_InDetAmbiguitySolverSLHC_49}
\includegraphics[width=.40\textwidth,height=.40\textheight,type=png,ext=.png,read=.png]{/afs/cern.ch/user/a/atpathak/afswork/public/Pixel/ALLTXT_6Oct2017_r9871_ID/pic/mydata_InDetAmbiguitySolverSLHC_49}

\end{center}
\end{normalsize}
\end{center}
\end{frame}
%-------------------------------------------------
\begin{frame}
\frametitle{Times/event (seconds) for Cluster Finding}
\begin{center}
\begin{normalsize}
Full Detector (left)  and Inner Detector (right) \\
\vspace*{-.2cm}
\begin{center}
\includegraphics[width=.40\textwidth,height=.40\textheight,type=png,ext=.png,read=.png]{/afs/cern.ch/user/a/atpathak/afswork/public/Pixel/ALLTXT_6Oct2017_r9871/pic/mydata_InDetBothClusterization_49}
\includegraphics[width=.40\textwidth,height=.40\textheight,type=png,ext=.png,read=.png]{/afs/cern.ch/user/a/atpathak/afswork/public/Pixel/ALLTXT_6Oct2017_r9871_ID/pic/mydata_InDetBothClusterization_49}

\end{center}
\end{normalsize}
\end{center}
\end{frame}
%-------------------------------------------------
\begin{frame}
\frametitle{Times/event (seconds) for Space Point Finder}
\begin{center}
\begin{normalsize}
Full Detector (left)  and Inner Detector (right) \\
\vspace*{-.2cm}
\begin{center}
\includegraphics[width=.40\textwidth,height=.40\textheight,type=png,ext=.png,read=.png]{/afs/cern.ch/user/a/atpathak/afswork/public/Pixel/ALLTXT_6Oct2017_r9871/pic/mydata_InDetSiTrackerSpacePointFinder_49}
\includegraphics[width=.40\textwidth,height=.40\textheight,type=png,ext=.png,read=.png]{/afs/cern.ch/user/a/atpathak/afswork/public/Pixel/ALLTXT_6Oct2017_r9871_ID/pic/mydata_InDetSiTrackerSpacePointFinder_49}

\end{center}
\end{normalsize}
\end{center}
\end{frame}
%-------------------------------------------------
\begin{frame}
\frametitle{Times/event (seconds) for Track Particles}
\begin{center}
\begin{normalsize}
Full Detector (left)  and Inner Detector (right) \\
\vspace*{-.2cm}
\begin{center}
\includegraphics[width=.40\textwidth,height=.40\textheight,type=png,ext=.png,read=.png]{/afs/cern.ch/user/a/atpathak/afswork/public/Pixel/ALLTXT_6Oct2017_r9871/pic/mydata_InDetTrackParticles_49}
\includegraphics[width=.40\textwidth,height=.40\textheight,type=png,ext=.png,read=.png]{/afs/cern.ch/user/a/atpathak/afswork/public/Pixel/ALLTXT_6Oct2017_r9871_ID/pic/mydata_InDetTrackParticles_49}

\end{center}
\end{normalsize}
\end{center}
\end{frame}
%-------------------------------------------------
\begin{frame}
\frametitle{Times/event (seconds) for Vertex Finder}
\begin{center}
\begin{normalsize}
Full Detector (left)  and Inner Detector (right) \\
\vspace*{-.2cm}
\begin{center}
\includegraphics[width=.40\textwidth,height=.40\textheight,type=png,ext=.png,read=.png]{/afs/cern.ch/user/a/atpathak/afswork/public/Pixel/ALLTXT_6Oct2017_r9871/pic/mydata_InDetPriVxFinder_49}
\includegraphics[width=.40\textwidth,height=.40\textheight,type=png,ext=.png,read=.png]{/afs/cern.ch/user/a/atpathak/afswork/public/Pixel/ALLTXT_6Oct2017_r9871_ID/pic/mydata_InDetPriVxFinder_49}

\end{center}
\end{normalsize}
\end{center}
\end{frame}
%-------------------------------------------------------
\begin{frame}
\frametitle{Summary of results (in HS06 seconds)}
\begin{itemize}
\item {\small These studies were done on Intel Xeon 7210, 1.3GHz 64-core CPU. The CPU time is multiplied with the HS06 factor of 3.5 /sec.}
\item Full Detector : 
\end{itemize}
\vspace*{-0.1cm} 
\begin{table}

{\scalebox{.45}{
\begin{tabular}{|l|c||c|c|c|c|c|c||c|}\hline\hline
                           
Detector&  Pile &              Cluster           &             Space                 &         Si Track                      &         Ambiguity                &          TRT + Back            &           Primary                 &        total              \\ 
             &  Up  &               Finding          &              Points                &         Finding                      &         Solution                    &           Tracking              &           Vertex                   &        ITk/ID            \\ \hline
Run-2   & 20   &  1.5 $\pm$ 0.1            &  0.74 $\pm$ 0.03           &    22 $\pm$ 1.1                 & 15.4 $\pm$ 0.7                &      1.13 $\pm$ 0.1         &     0.51 $\pm$ 0.02        &                              \\\hline
ITk        & 0     &  0.44 $\pm$ 0.02        &  0.15 $\pm$ 0.01           &   0.66 $\pm$ 0.03             &  8.5 $\pm$ 0.4                 &               $-$                   &     0.40 $\pm$ 0.02        &                              \\\hline
ITk        & 60   &  7 $\pm$ 0.3               &  3.4 $\pm$ 0.2               &   11.0 $\pm$ 0.5               &  75 $\pm$ 3                     &               $-$                   &     1.7 $\pm$ 0.1            &                              \\\hline
ITk        & 140 &  16 $\pm$ 0.7             &  12 $\pm$ 0.6                &   52 $\pm$ 2                     &  173 $\pm$ 8                   &               $-$                   &    4 $\pm$ 0.2                &                              \\\hline
ITk        & 200 &  25.3 $\pm$ 1.1          &  23 $\pm$ 1.1                &   123 $\pm$ 5                   &  260 $\pm$ 11                 &               $-$                   &     6 $\pm$ 0.3               &                              \\\hline

%ITk        & 140 &  16.2$\pm$ 0.71        &  12.2 $\pm$0.58             &   51.6$\pm$ 2.25              &  173.2$\pm$ 7.51           &               $-$                   &    4.17$\pm$0.189        &                              \\\hline
%ITk        & 60   &  6.7$\pm$ 0.29          &  3.4 $\pm$0.15               &   11.0$\pm$ 0.49              &  75.1$\pm$ 3.28             &               $-$                   &     1.7$\pm$0.08        &                              \\\hline
%ITk        & 0     &  0.44$\pm$ 0.02        &  0.15 $\pm$0.007           &   0.66$\pm$ 0.03              &  8.50$\pm$ 0.39             &               $-$                   &     0.40$\pm$0.021        &                              \\\hline
%Run-2   & 20   &  1.46$\pm$0.066       &  0.74$\pm$0.034            &    22.1$\pm$1.097             & 15.4 $\pm$ 0.74            &      1.13$\pm$0.066       &     0.51 $\pm$0.023        &                              \\\hline
\hline
\end{tabular}
}}
\end{table}

\vspace*{-0.1cm} 
\begin{itemize}
\item Inner Detector only : 
\end{itemize}
\begin{table}
{\scalebox{.45}{
\begin{tabular}{|l|c||c|c|c|c|c|c||c|}\hline\hline
                           
Detector&  Pile &              Cluster           &              Space                  &         Si Track                       &         Ambiguity               &          TRT + Back              &           Primary                &        total              \\ 
             &  Up  &               Finding          &              Points                 &         Finding                        &         Solution                  &           Tracking                 &           Vertex                  &        ITk/ID            \\ \hline
Run-2   & 20   &  1.5 $\pm$ 0.1            &  0.72 $\pm$ 0.03            &    21 $\pm$ 1.1                   & 14 $\pm$ 0.7                  &        0.1 $\pm$ 0.003      &     0.48 $\pm$ 0.02        &                              \\\hline
ITk        & 0     &  0.43 $\pm$ 0.02        &  0.14 $\pm$ 0.01            &   0.65 $\pm$ 0.03               & 8 $\pm$ 0.4                    &               $-$                    &    0.39 $\pm$ 0.02         &                              \\\hline
ITk        & 60   &  6.5 $\pm$ 0.3            &  3.4 $\pm$ 0.2                &   11.3 $\pm$ 0.5                 &  73 $\pm$ 3                    &               $-$                    &     1.7 $\pm$ 0.1            &                              \\\hline
ITk        & 140 &  16 $\pm$ 0.7             &  12.4 $\pm$ 0.6              &   58 $\pm$ 3                       &  171 $\pm$ 7                  &               $-$                    &    4 $\pm$ 0.2                &                              \\\hline
ITk        & 200 &  24.3 $\pm$ 1.0          &  21.5 $\pm$ 0.9              &   138 $\pm$ 6                     &  256 $\pm$ 11                &               $-$                    &     5.6 $\pm$ 0.2            &                              \\\hline
\hline
\end{tabular}
}}
\end{table}

\vspace*{0.5cm} 

\begin{itemize}
\item {\small List of Algorithm Names :} 
\end{itemize}
\vspace*{-0.5cm} 
\begin{table}

{\scalebox{.33}{
\begin{tabular}{|l|c||c|c|c|c|c|c||c|}\hline\hline
                           
Detector&  Pile &              Cluster                   &             Space                            &         Si Track                      &         Ambiguity               &          TRT + Back            &           Primary                 &        total              \\ 
             &  Up  &               Finding                   &             Points                           &         Finding                      &         Solution                   &           Tracking               &           Vertex                  &        ITk/ID            \\ \hline
ITk        & 200 &  InDetSCT\_Clusterization + &InDetSiTrackerSpacePointFinder&InDetSiSpTrackFinderSLHC &InDetAmbiguitySolverSLHC&               $-$                  &InDetPriVxFinder            &                              \\
             &        &  InDetPixelClusterization        &                                                &                                           &                                           &               $-$                   &                                     &                              \\\hline
Run-2   & 20 &  InDetSCT\_Clusterization + &InDetSiTrackerSpacePointFinder&InDetSiSpTrackFinder          &InDetAmbiguitySolver +                    & InDetTRT\_TrackSegmentsFinder+&InDetPriVxFinder&           \\
             &        &  InDetPixelClusterization        &                                                &                                           & InDetAmbiguitySolverForwardTracks&InDetTRT\_SeededTrackFinder&    &                              \\\hline

\hline
\end{tabular}
}}
\end{table}
\vspace*{1cm}
\end{frame}

%-------------------------------------------------
\begin{frame}
\frametitle{Benchmarking according to HEPSPEC06 for Full Detector}

\begin{center}
\begin{normalsize}

%\begin{itemize}

%\item Time (minute) / event 
%\item Timing for r9060 [left], r9133 [center] and r9139 [right]\\
%\item \\
%\item \\
\vspace*{-.2cm}
\begin{center}
%\item For r9060(left) and For r9133(right) \\
\includegraphics[width=.65\textwidth,height=.95\textheight,type=pdf,ext=.pdf,read=.pdf]{/afs/cern.ch/user/a/atpathak/afswork/public/Pixel/ALLTXT_6Oct2017_r9871/All_PileUp_log_new}
\end{center}
%\item \\
%\end{itemize}
\end{normalsize}
\end{center}
\end{frame}
%-------------------------------------------------
\begin{frame}
\frametitle{Benchmarking according to HEPSPEC06 for Inner Detector}

\begin{center}
\begin{normalsize}

%\begin{itemize}

%\item Time (minute) / event 
%\item Timing for r9060 [left], r9133 [center] and r9139 [right]\\
%\item \\
%\item \\
\vspace*{-.2cm}
\begin{center}
%\item For r9060(left) and For r9133(right) \\
\includegraphics[width=.65\textwidth,height=.95\textheight,type=pdf,ext=.pdf,read=.pdf]{/afs/cern.ch/user/a/atpathak/afswork/public/Pixel/ALLTXT_6Oct2017_r9871_ID/All_PileUp_log_new}
\end{center}
%\item \\
%\end{itemize}
\end{normalsize}
\end{center}
\end{frame}
%-------------------------------------------------
\begin{frame}
\frametitle{Linearity Test for r9871}

\begin{normalsize}
\begin{itemize}
\vspace*{-.5cm}
\item 20 jobs run over for 1, 2, 5, 10, 20, 50, 100 events \\
%\item Number events vs Fitted Mean of AthAlgSeq 
%\vspace*{-.1cm}
\begin{center}
\includegraphics[width=.45\textwidth,height=.45\textheight,type=png,ext=.png,read=.png]{/afs/cern.ch/user/a/atpathak/afswork/public/Pixel/ALLTXT_9Oct2017_r9871/AthAlgSeq_LinearFit_0} 
\includegraphics[width=.45\textwidth,height=.45\textheight,type=png,ext=.png,read=.png]{/afs/cern.ch/user/a/atpathak/afswork/public/Pixel/ALLTXT_9Oct2017_r9871/AthAlgSeq_LinearFit_1}
\end{center}
%\item Number events vs Fitted Mean per event of AthAlgSeq

\end{itemize}
\end{normalsize}

\end{frame}
%-------------------------------------------------
\begin{frame}
\frametitle{Stability Test for r9871}

\begin{normalsize}
\begin{itemize}
\vspace*{-.5cm}
\item 50 events run over in 10, 15, 20, 25, 30 jobs \\
%\item Number events vs Fitted Mean of AthAlgSeq 
%\vspace*{-.1cm}
\begin{center}
\includegraphics[width=.45\textwidth,height=.45\textheight,type=png,ext=.png,read=.png]{/afs/cern.ch/user/a/atpathak/afswork/public/Pixel/ALLTXT_11Oct2017_r9871/AthAlgSeq_LinearFit_1} 

\end{center}
%\item Number events vs Fitted Mean per event of AthAlgSeq

\end{itemize}
\end{normalsize}

\end{frame}
\end{document}
%---------------------------------------
\begin{frame}
\frametitle{Times/event (seconds) for AthAlgSeq}
\begin{center}
\begin{normalsize}
Full Detector (left)  and Inner Detectro(right) \\
\vspace*{-.2cm}
\begin{center}
\includegraphics[width=.40\textwidth,height=.40\textheight,type=png,ext=.png,read=.png]{/afs/cern.ch/user/a/atpathak/afswork/public/Pixel/ALLTXT_6Oct2017_r7725/pic/mydata_AthAlgSeq_49}
\includegraphics[width=.40\textwidth,height=.40\textheight,type=png,ext=.png,read=.png]{/afs/cern.ch/user/a/atpathak/afswork/public/Pixel/ALLTXT_6Oct2017_r7725_ID/pic/mydata_AthAlgSeq_49}

\end{center}
\end{normalsize}
\end{center}
\end{frame}
%-------------------------------------------------
\begin{frame}
\frametitle{Times/event (seconds) for InDetSiSpTrackFinder}
\begin{center}
\begin{normalsize}
Full Detector (left)  and Inner Detectro(right) \\
\vspace*{-.2cm}
\begin{center}
\includegraphics[width=.40\textwidth,height=.40\textheight,type=png,ext=.png,read=.png]{/afs/cern.ch/user/a/atpathak/afswork/public/Pixel/ALLTXT_6Oct2017_r7725/pic/mydata_InDetSiSpTrackFinder_49}
\includegraphics[width=.40\textwidth,height=.40\textheight,type=png,ext=.png,read=.png]{/afs/cern.ch/user/a/atpathak/afswork/public/Pixel/ALLTXT_6Oct2017_r7725_ID/pic/mydata_InDetSiSpTrackFinder_49}

\end{center}
\end{normalsize}
\end{center}
\end{frame}
%-------------------------------------------------
\begin{frame}
\frametitle{Times/event (seconds) for InDetAmbiguitySolver}
\begin{center}
\begin{normalsize}
Full Detector (left)  and Inner Detectro(right) \\
\vspace*{-.2cm}
\begin{center}
\includegraphics[width=.40\textwidth,height=.40\textheight,type=png,ext=.png,read=.png]{/afs/cern.ch/user/a/atpathak/afswork/public/Pixel/ALLTXT_6Oct2017_r7725/pic/mydata_InDetAmbiguitySolver_49}
\includegraphics[width=.40\textwidth,height=.40\textheight,type=png,ext=.png,read=.png]{/afs/cern.ch/user/a/atpathak/afswork/public/Pixel/ALLTXT_6Oct2017_r7725_ID/pic/mydata_InDetAmbiguitySolver_49}

\end{center}
\end{normalsize}
\end{center}
\end{frame}
%-------------------------------------------------
\begin{frame}
\frametitle{Times/event (seconds) for InDetAmbiguitySolverForwardTracks + InDetAmbiguitySolver}
\begin{center}
\begin{normalsize}
Full Detector (left)  and Inner Detectro(right) \\
\vspace*{-.2cm}
\begin{center}
\includegraphics[width=.40\textwidth,height=.40\textheight,type=png,ext=.png,read=.png]{/afs/cern.ch/user/a/atpathak/afswork/public/Pixel/ALLTXT_6Oct2017_r7725/pic/mydata_InDetBothAmbiguitySolver_49}
\includegraphics[width=.40\textwidth,height=.40\textheight,type=png,ext=.png,read=.png]{/afs/cern.ch/user/a/atpathak/afswork/public/Pixel/ALLTXT_6Oct2017_r7725_ID/pic/mydata_InDetBothAmbiguitySolver_49}

\end{center}
\end{normalsize}
\end{center}
\end{frame}
%-------------------------------------------------
\begin{frame}
\frametitle{Times/event (seconds) for ForwardTrackingg}
\begin{center}
\begin{normalsize}
Full Detector (left)  and Inner Detectro(right) \\
\vspace*{-.2cm}
\begin{center}
\includegraphics[width=.40\textwidth,height=.40\textheight,type=png,ext=.png,read=.png]{/afs/cern.ch/user/a/atpathak/afswork/public/Pixel/ALLTXT_6Oct2017_r7725/pic/mydata_InDetBothForwardTracking_49}
\includegraphics[width=.40\textwidth,height=.40\textheight,type=png,ext=.png,read=.png]{/afs/cern.ch/user/a/atpathak/afswork/public/Pixel/ALLTXT_6Oct2017_r7725_ID/pic/mydata_InDetBothForwardTracking_49}

\end{center}
\end{normalsize}
\end{center}
\end{frame}
%-------------------------------------------------
\begin{frame}
\frametitle{Times/event (seconds) for Back tracking}
\begin{center}
\begin{normalsize}
Full Detector (left)  and Inner Detectro(right) \\
\vspace*{-.2cm}
\begin{center}
\includegraphics[width=.40\textwidth,height=.40\textheight,type=png,ext=.png,read=.png]{/afs/cern.ch/user/a/atpathak/afswork/public/Pixel/ALLTXT_6Oct2017_r7725/pic/mydata_InDetBothBackTracking_49}
\includegraphics[width=.40\textwidth,height=.40\textheight,type=png,ext=.png,read=.png]{/afs/cern.ch/user/a/atpathak/afswork/public/Pixel/ALLTXT_6Oct2017_r7725_ID/pic/mydata_InDetBothBackTracking_49}

\end{center}
\end{normalsize}
\end{center}
\end{frame}
%-------------------------------------------------
\begin{frame}
\frametitle{Times/event (seconds) for BothClusterization}
\begin{center}
\begin{normalsize}
Full Detector (left)  and Inner Detectro(right) \\
\vspace*{-.2cm}
\begin{center}
\includegraphics[width=.40\textwidth,height=.40\textheight,type=png,ext=.png,read=.png]{/afs/cern.ch/user/a/atpathak/afswork/public/Pixel/ALLTXT_6Oct2017_r7725/pic/mydata_InDetBothClusterization_49}
\includegraphics[width=.40\textwidth,height=.40\textheight,type=png,ext=.png,read=.png]{/afs/cern.ch/user/a/atpathak/afswork/public/Pixel/ALLTXT_6Oct2017_r7725_ID/pic/mydata_InDetBothClusterization_49}

\end{center}
\end{normalsize}
\end{center}
\end{frame}
%-------------------------------------------------
\begin{frame}
\frametitle{Times/event (seconds) for InDetSiTrackerSpacePointFinder}
\begin{center}
\begin{normalsize}
Full Detector (left)  and Inner Detectro(right) \\
\vspace*{-.2cm}
\begin{center}
\includegraphics[width=.40\textwidth,height=.40\textheight,type=png,ext=.png,read=.png]{/afs/cern.ch/user/a/atpathak/afswork/public/Pixel/ALLTXT_6Oct2017_r7725/pic/mydata_InDetSiTrackerSpacePointFinder_49}
\includegraphics[width=.40\textwidth,height=.40\textheight,type=png,ext=.png,read=.png]{/afs/cern.ch/user/a/atpathak/afswork/public/Pixel/ALLTXT_6Oct2017_r7725_ID/pic/mydata_InDetSiTrackerSpacePointFinder_49}

\end{center}
\end{normalsize}
\end{center}
\end{frame}
%-------------------------------------------------
\begin{frame}
\frametitle{Times/event (seconds) for InDetPriVxFinder}
\begin{center}
\begin{normalsize}
Full Detector (left)  and Inner Detectro(right) \\
\vspace*{-.2cm}
\begin{center}
\includegraphics[width=.40\textwidth,height=.40\textheight,type=png,ext=.png,read=.png]{/afs/cern.ch/user/a/atpathak/afswork/public/Pixel/ALLTXT_6Oct2017_r7725/pic/mydata_InDetPriVxFinder_49}
\includegraphics[width=.40\textwidth,height=.40\textheight,type=png,ext=.png,read=.png]{/afs/cern.ch/user/a/atpathak/afswork/public/Pixel/ALLTXT_6Oct2017_r7725_ID/pic/mydata_InDetPriVxFinder_49}

\end{center}
\end{normalsize}
\end{center}
\end{frame}
%------------------------------------------------
\begin{frame}
\begin{Large}
\centerline{Reconstruction times for ITk studies r9871}
\end{Large}
\end{frame}
%-------------------------------------------------
\begin{frame}
\frametitle{Times/event (seconds) for AthAlgSeq}
\begin{center}
\begin{normalsize}
Full Detector (left)  and Inner Detectro(right) \\
\vspace*{-.2cm}
\begin{center}
\includegraphics[width=.40\textwidth,height=.40\textheight,type=png,ext=.png,read=.png]{/afs/cern.ch/user/a/atpathak/afswork/public/Pixel/ALLTXT_6Oct2017_r9871/pic/mydata_AthAlgSeq_49}
\includegraphics[width=.40\textwidth,height=.40\textheight,type=png,ext=.png,read=.png]{/afs/cern.ch/user/a/atpathak/afswork/public/Pixel/ALLTXT_6Oct2017_r9871_ID/pic/mydata_AthAlgSeq_49}

\end{center}
\end{normalsize}
\end{center}
\end{frame}
%-------------------------------------------------
\begin{frame}
\frametitle{Times/event (seconds) for InDetSiSpTrackFinderSLHC}
\begin{center}
\begin{normalsize}
Full Detector (left)  and Inner Detectro(right) \\
\vspace*{-.2cm}
\begin{center}
\includegraphics[width=.40\textwidth,height=.40\textheight,type=png,ext=.png,read=.png]{/afs/cern.ch/user/a/atpathak/afswork/public/Pixel/ALLTXT_6Oct2017_r9871/pic/mydata_InDetSiSpTrackFinderSLHC_49}
\includegraphics[width=.40\textwidth,height=.40\textheight,type=png,ext=.png,read=.png]{/afs/cern.ch/user/a/atpathak/afswork/public/Pixel/ALLTXT_6Oct2017_r9871_ID/pic/mydata_InDetSiSpTrackFinderSLHC_49}

\end{center}
\end{normalsize}
\end{center}
\end{frame}
%-------------------------------------------------
\begin{frame}
\frametitle{Times/event (seconds) for InDetAmbiguitySolverSLHC}
\begin{center}
\begin{normalsize}
Full Detector (left)  and Inner Detectro(right) \\
\vspace*{-.2cm}
\begin{center}
\includegraphics[width=.40\textwidth,height=.40\textheight,type=png,ext=.png,read=.png]{/afs/cern.ch/user/a/atpathak/afswork/public/Pixel/ALLTXT_6Oct2017_r9871/pic/mydata_InDetAmbiguitySolverSLHC_49}
\includegraphics[width=.40\textwidth,height=.40\textheight,type=png,ext=.png,read=.png]{/afs/cern.ch/user/a/atpathak/afswork/public/Pixel/ALLTXT_6Oct2017_r9871_ID/pic/mydata_InDetAmbiguitySolverSLHC_49}

\end{center}
\end{normalsize}
\end{center}
\end{frame}
%-------------------------------------------------
\begin{frame}
\frametitle{Times/event (seconds) for BothClusterization}
\begin{center}
\begin{normalsize}
Full Detector (left)  and Inner Detectro(right) \\
\vspace*{-.2cm}
\begin{center}
\includegraphics[width=.40\textwidth,height=.40\textheight,type=png,ext=.png,read=.png]{/afs/cern.ch/user/a/atpathak/afswork/public/Pixel/ALLTXT_6Oct2017_r9871/pic/mydata_InDetBothClusterization_49}
\includegraphics[width=.40\textwidth,height=.40\textheight,type=png,ext=.png,read=.png]{/afs/cern.ch/user/a/atpathak/afswork/public/Pixel/ALLTXT_6Oct2017_r9871_ID/pic/mydata_InDetBothClusterization_49}

\end{center}
\end{normalsize}
\end{center}
\end{frame}
%-------------------------------------------------
\begin{frame}
\frametitle{Times/event (seconds) for InDetSiTrackerSpacePointFinder}
\begin{center}
\begin{normalsize}
Full Detector (left)  and Inner Detectro(right) \\
\vspace*{-.2cm}
\begin{center}
\includegraphics[width=.40\textwidth,height=.40\textheight,type=png,ext=.png,read=.png]{/afs/cern.ch/user/a/atpathak/afswork/public/Pixel/ALLTXT_6Oct2017_r9871/pic/mydata_InDetSiTrackerSpacePointFinder_49}
\includegraphics[width=.40\textwidth,height=.40\textheight,type=png,ext=.png,read=.png]{/afs/cern.ch/user/a/atpathak/afswork/public/Pixel/ALLTXT_6Oct2017_r9871_ID/pic/mydata_InDetSiTrackerSpacePointFinder_49}

\end{center}
\end{normalsize}
\end{center}
\end{frame}
%-------------------------------------------------
\begin{frame}
\frametitle{Times/event (seconds) for InDetTrackParticles}
\begin{center}
\begin{normalsize}
Full Detector (left)  and Inner Detectro(right) \\
\vspace*{-.2cm}
\begin{center}
\includegraphics[width=.40\textwidth,height=.40\textheight,type=png,ext=.png,read=.png]{/afs/cern.ch/user/a/atpathak/afswork/public/Pixel/ALLTXT_6Oct2017_r9871/pic/mydata_InDetTrackParticles_49}
\includegraphics[width=.40\textwidth,height=.40\textheight,type=png,ext=.png,read=.png]{/afs/cern.ch/user/a/atpathak/afswork/public/Pixel/ALLTXT_6Oct2017_r9871_ID/pic/mydata_InDetTrackParticles_49}

\end{center}
\end{normalsize}
\end{center}
\end{frame}
%-------------------------------------------------
\begin{frame}
\frametitle{Times/event (seconds) for InDetPriVxFinder}
\begin{center}
\begin{normalsize}
Full Detector (left)  and Inner Detectro(right) \\
\vspace*{-.2cm}
\begin{center}
\includegraphics[width=.40\textwidth,height=.40\textheight,type=png,ext=.png,read=.png]{/afs/cern.ch/user/a/atpathak/afswork/public/Pixel/ALLTXT_6Oct2017_r9871/pic/mydata_InDetPriVxFinder_49}
\includegraphics[width=.40\textwidth,height=.40\textheight,type=png,ext=.png,read=.png]{/afs/cern.ch/user/a/atpathak/afswork/public/Pixel/ALLTXT_6Oct2017_r9871_ID/pic/mydata_InDetPriVxFinder_49}

\end{center}
\end{normalsize}
\end{center}
\end{frame}
%-------------------------------------------------
\begin{frame}
\frametitle{Times/event (seconds) for InDetVxLinkSetter}
\begin{center}
\begin{normalsize}
Full Detector (left)  and Inner Detectro(right) \\
\vspace*{-.2cm}
\begin{center}
\includegraphics[width=.40\textwidth,height=.40\textheight,type=png,ext=.png,read=.png]{/afs/cern.ch/user/a/atpathak/afswork/public/Pixel/ALLTXT_6Oct2017_r9871/pic/mydata_InDetVxLinkSetter_49}
\includegraphics[width=.40\textwidth,height=.40\textheight,type=png,ext=.png,read=.png]{/afs/cern.ch/user/a/atpathak/afswork/public/Pixel/ALLTXT_6Oct2017_r9871_ID/pic/mydata_InDetVxLinkSetter_49}

\end{center}
\end{normalsize}
\end{center}
\end{frame}
%-------------------------------------------------
\begin{frame}
\frametitle{Times/event (seconds) for InDetRecStatistics}
\begin{center}
\begin{normalsize}
Full Detector (left)  and Inner Detectro(right) \\
\vspace*{-.2cm}
\begin{center}
\includegraphics[width=.40\textwidth,height=.40\textheight,type=png,ext=.png,read=.png]{/afs/cern.ch/user/a/atpathak/afswork/public/Pixel/ALLTXT_6Oct2017_r9871/pic/mydata_InDetRecStatistics_49}
\includegraphics[width=.40\textwidth,height=.40\textheight,type=png,ext=.png,read=.png]{/afs/cern.ch/user/a/atpathak/afswork/public/Pixel/ALLTXT_6Oct2017_r9871_ID/pic/mydata_InDetRecStatistics_49}

\end{center}
\end{normalsize}
\end{center}
\end{frame}
%-------------------------------------------------
\begin{frame}
\frametitle{Times/event (seconds) for InDetCaloClusterROISelector}
\begin{center}
\begin{normalsize}
Full Detector (left)  and Inner Detectro(right) \\
\vspace*{-.2cm}
\begin{center}
\includegraphics[width=.40\textwidth,height=.40\textheight,type=png,ext=.png,read=.png]{/afs/cern.ch/user/a/atpathak/afswork/public/Pixel/ALLTXT_6Oct2017_r9871/pic/mydata_InDetCaloClusterROISelector_49}
%\includegraphics[width=.40\textwidth,height=.40\textheight,type=png,ext=.png,read=.png]{/afs/cern.ch/user/a/atpathak/afswork/public/Pixel/ALLTXT_6Oct2017_r9871_ID/pic/mydata_InDetCaloClusterROISelector_49}

\end{center}
\end{normalsize}
\end{center}
\end{frame}
%------------------------------------------------
\begin{frame}
\begin{Large}
\centerline{Reconstruction times for ITk studies r9875}
\end{Large}
\end{frame}
%-------------------------------------------------
\begin{frame}
\frametitle{Times/event (seconds) for AthAlgSeq}
\begin{center}
\begin{normalsize}
Full Detector (left)  and Inner Detectro(right) \\
\vspace*{-.2cm}
\begin{center}
\includegraphics[width=.40\textwidth,height=.40\textheight,type=png,ext=.png,read=.png]{/afs/cern.ch/user/a/atpathak/afswork/public/Pixel/ALLTXT_6Oct2017_r9875/pic/mydata_AthAlgSeq_49}
\includegraphics[width=.40\textwidth,height=.40\textheight,type=png,ext=.png,read=.png]{/afs/cern.ch/user/a/atpathak/afswork/public/Pixel/ALLTXT_6Oct2017_r9875_ID/pic/mydata_AthAlgSeq_49}

\end{center}
\end{normalsize}
\end{center}
\end{frame}
%-------------------------------------------------
\begin{frame}
\frametitle{Times/event (seconds) for InDetSiSpTrackFinderSLHC}
\begin{center}
\begin{normalsize}
Full Detector (left)  and Inner Detectro(right) \\
\vspace*{-.2cm}
\begin{center}
\includegraphics[width=.40\textwidth,height=.40\textheight,type=png,ext=.png,read=.png]{/afs/cern.ch/user/a/atpathak/afswork/public/Pixel/ALLTXT_6Oct2017_r9875/pic/mydata_InDetSiSpTrackFinderSLHC_49}
\includegraphics[width=.40\textwidth,height=.40\textheight,type=png,ext=.png,read=.png]{/afs/cern.ch/user/a/atpathak/afswork/public/Pixel/ALLTXT_6Oct2017_r9875_ID/pic/mydata_InDetSiSpTrackFinderSLHC_49}

\end{center}
\end{normalsize}
\end{center}
\end{frame}
%-------------------------------------------------
\begin{frame}
\frametitle{Times/event (seconds) for InDetAmbiguitySolverSLHC}
\begin{center}
\begin{normalsize}
Full Detector (left)  and Inner Detectro(right) \\
\vspace*{-.2cm}
\begin{center}
\includegraphics[width=.40\textwidth,height=.40\textheight,type=png,ext=.png,read=.png]{/afs/cern.ch/user/a/atpathak/afswork/public/Pixel/ALLTXT_6Oct2017_r9875/pic/mydata_InDetAmbiguitySolverSLHC_49}
\includegraphics[width=.40\textwidth,height=.40\textheight,type=png,ext=.png,read=.png]{/afs/cern.ch/user/a/atpathak/afswork/public/Pixel/ALLTXT_6Oct2017_r9875_ID/pic/mydata_InDetAmbiguitySolverSLHC_49}

\end{center}
\end{normalsize}
\end{center}
\end{frame}
%-------------------------------------------------
\begin{frame}
\frametitle{Times/event (seconds) for BothClusterization}
\begin{center}
\begin{normalsize}
Full Detector (left)  and Inner Detectro(right) \\
\vspace*{-.2cm}
\begin{center}
\includegraphics[width=.40\textwidth,height=.40\textheight,type=png,ext=.png,read=.png]{/afs/cern.ch/user/a/atpathak/afswork/public/Pixel/ALLTXT_6Oct2017_r9875/pic/mydata_InDetBothClusterization_49}
\includegraphics[width=.40\textwidth,height=.40\textheight,type=png,ext=.png,read=.png]{/afs/cern.ch/user/a/atpathak/afswork/public/Pixel/ALLTXT_6Oct2017_r9875_ID/pic/mydata_InDetBothClusterization_49}

\end{center}
\end{normalsize}
\end{center}
\end{frame}
%-------------------------------------------------
\begin{frame}
\frametitle{Times/event (seconds) for InDetSiTrackerSpacePointFinder}
\begin{center}
\begin{normalsize}
Full Detector (left)  and Inner Detectro(right) \\
\vspace*{-.2cm}
\begin{center}
\includegraphics[width=.40\textwidth,height=.40\textheight,type=png,ext=.png,read=.png]{/afs/cern.ch/user/a/atpathak/afswork/public/Pixel/ALLTXT_6Oct2017_r9875/pic/mydata_InDetSiTrackerSpacePointFinder_49}
\includegraphics[width=.40\textwidth,height=.40\textheight,type=png,ext=.png,read=.png]{/afs/cern.ch/user/a/atpathak/afswork/public/Pixel/ALLTXT_6Oct2017_r9875_ID/pic/mydata_InDetSiTrackerSpacePointFinder_49}

\end{center}
\end{normalsize}
\end{center}
\end{frame}
%-------------------------------------------------
\begin{frame}
\frametitle{Times/event (seconds) for InDetTrackParticles}
\begin{center}
\begin{normalsize}
Full Detector (left)  and Inner Detectro(right) \\
\vspace*{-.2cm}
\begin{center}
\includegraphics[width=.40\textwidth,height=.40\textheight,type=png,ext=.png,read=.png]{/afs/cern.ch/user/a/atpathak/afswork/public/Pixel/ALLTXT_6Oct2017_r9875/pic/mydata_InDetTrackParticles_49}
\includegraphics[width=.40\textwidth,height=.40\textheight,type=png,ext=.png,read=.png]{/afs/cern.ch/user/a/atpathak/afswork/public/Pixel/ALLTXT_6Oct2017_r9875_ID/pic/mydata_InDetTrackParticles_49}

\end{center}
\end{normalsize}
\end{center}
\end{frame}
%-------------------------------------------------
\begin{frame}
\frametitle{Times/event (seconds) for InDetPriVxFinder}
\begin{center}
\begin{normalsize}
Full Detector (left)  and Inner Detectro(right) \\
\vspace*{-.2cm}
\begin{center}
\includegraphics[width=.40\textwidth,height=.40\textheight,type=png,ext=.png,read=.png]{/afs/cern.ch/user/a/atpathak/afswork/public/Pixel/ALLTXT_6Oct2017_r9875/pic/mydata_InDetPriVxFinder_49}
\includegraphics[width=.40\textwidth,height=.40\textheight,type=png,ext=.png,read=.png]{/afs/cern.ch/user/a/atpathak/afswork/public/Pixel/ALLTXT_6Oct2017_r9875_ID/pic/mydata_InDetPriVxFinder_49}

\end{center}
\end{normalsize}
\end{center}
\end{frame}
%-------------------------------------------------
\begin{frame}
\frametitle{Times/event (seconds) for InDetVxLinkSetter}
\begin{center}
\begin{normalsize}
Full Detector (left)  and Inner Detectro(right) \\
\vspace*{-.2cm}
\begin{center}
\includegraphics[width=.40\textwidth,height=.40\textheight,type=png,ext=.png,read=.png]{/afs/cern.ch/user/a/atpathak/afswork/public/Pixel/ALLTXT_6Oct2017_r9875/pic/mydata_InDetVxLinkSetter_49}
\includegraphics[width=.40\textwidth,height=.40\textheight,type=png,ext=.png,read=.png]{/afs/cern.ch/user/a/atpathak/afswork/public/Pixel/ALLTXT_6Oct2017_r9875_ID/pic/mydata_InDetVxLinkSetter_49}

\end{center}
\end{normalsize}
\end{center}
\end{frame}
%-------------------------------------------------
\begin{frame}
\frametitle{Times/event (seconds) for InDetRecStatistics}
\begin{center}
\begin{normalsize}
Full Detector (left)  and Inner Detectro(right) \\
\vspace*{-.2cm}
\begin{center}
\includegraphics[width=.40\textwidth,height=.40\textheight,type=png,ext=.png,read=.png]{/afs/cern.ch/user/a/atpathak/afswork/public/Pixel/ALLTXT_6Oct2017_r9875/pic/mydata_InDetRecStatistics_49}
\includegraphics[width=.40\textwidth,height=.40\textheight,type=png,ext=.png,read=.png]{/afs/cern.ch/user/a/atpathak/afswork/public/Pixel/ALLTXT_6Oct2017_r9875_ID/pic/mydata_InDetRecStatistics_49}

\end{center}
\end{normalsize}
\end{center}
\end{frame}
%-------------------------------------------------
\begin{frame}
\frametitle{Times/event (seconds) for InDetCaloClusterROISelector}
\begin{center}
\begin{normalsize}
Full Detector (left)  and Inner Detectro(right) \\
\vspace*{-.2cm}
\begin{center}
\includegraphics[width=.40\textwidth,height=.40\textheight,type=png,ext=.png,read=.png]{/afs/cern.ch/user/a/atpathak/afswork/public/Pixel/ALLTXT_6Oct2017_r9875/pic/mydata_InDetCaloClusterROISelector_49}
%\includegraphics[width=.40\textwidth,height=.40\textheight,type=png,ext=.png,read=.png]{/afs/cern.ch/user/a/atpathak/afswork/public/Pixel/ALLTXT_6Oct2017_r9875_ID/pic/mydata_InDetCaloClusterROISelector_49}

\end{center}
\end{normalsize}
\end{center}
\end{frame}
%------------------------------------------------
\begin{frame}
\begin{Large}
\centerline{Reconstruction times for ITk studies r9876}
\end{Large}
\end{frame}
%-------------------------------------------------
\begin{frame}
\frametitle{Times/event (seconds) for AthAlgSeq}
\begin{center}
\begin{normalsize}
Full Detector (left)  and Inner Detectro(right) \\
\vspace*{-.2cm}
\begin{center}
\includegraphics[width=.40\textwidth,height=.40\textheight,type=png,ext=.png,read=.png]{/afs/cern.ch/user/a/atpathak/afswork/public/Pixel/ALLTXT_6Oct2017_r9876/pic/mydata_AthAlgSeq_49}
\includegraphics[width=.40\textwidth,height=.40\textheight,type=png,ext=.png,read=.png]{/afs/cern.ch/user/a/atpathak/afswork/public/Pixel/ALLTXT_6Oct2017_r9876_ID/pic/mydata_AthAlgSeq_49}

\end{center}
\end{normalsize}
\end{center}
\end{frame}
%-------------------------------------------------
\begin{frame}
\frametitle{Times/event (seconds) for InDetSiSpTrackFinderSLHC}
\begin{center}
\begin{normalsize}
Full Detector (left)  and Inner Detectro(right) \\
\vspace*{-.2cm}
\begin{center}
\includegraphics[width=.40\textwidth,height=.40\textheight,type=png,ext=.png,read=.png]{/afs/cern.ch/user/a/atpathak/afswork/public/Pixel/ALLTXT_6Oct2017_r9876/pic/mydata_InDetSiSpTrackFinderSLHC_49}
\includegraphics[width=.40\textwidth,height=.40\textheight,type=png,ext=.png,read=.png]{/afs/cern.ch/user/a/atpathak/afswork/public/Pixel/ALLTXT_6Oct2017_r9876_ID/pic/mydata_InDetSiSpTrackFinderSLHC_49}

\end{center}
\end{normalsize}
\end{center}
\end{frame}
%-------------------------------------------------
\begin{frame}
\frametitle{Times/event (seconds) for InDetAmbiguitySolverSLHC}
\begin{center}
\begin{normalsize}
Full Detector (left)  and Inner Detectro(right) \\
\vspace*{-.2cm}
\begin{center}
\includegraphics[width=.40\textwidth,height=.40\textheight,type=png,ext=.png,read=.png]{/afs/cern.ch/user/a/atpathak/afswork/public/Pixel/ALLTXT_6Oct2017_r9876/pic/mydata_InDetAmbiguitySolverSLHC_49}
\includegraphics[width=.40\textwidth,height=.40\textheight,type=png,ext=.png,read=.png]{/afs/cern.ch/user/a/atpathak/afswork/public/Pixel/ALLTXT_6Oct2017_r9876_ID/pic/mydata_InDetAmbiguitySolverSLHC_49}

\end{center}
\end{normalsize}
\end{center}
\end{frame}
%-------------------------------------------------
\begin{frame}
\frametitle{Times/event (seconds) for BothClusterization}
\begin{center}
\begin{normalsize}
Full Detector (left)  and Inner Detectro(right) \\
\vspace*{-.2cm}
\begin{center}
\includegraphics[width=.40\textwidth,height=.40\textheight,type=png,ext=.png,read=.png]{/afs/cern.ch/user/a/atpathak/afswork/public/Pixel/ALLTXT_6Oct2017_r9876/pic/mydata_InDetBothClusterization_49}
\includegraphics[width=.40\textwidth,height=.40\textheight,type=png,ext=.png,read=.png]{/afs/cern.ch/user/a/atpathak/afswork/public/Pixel/ALLTXT_6Oct2017_r9876_ID/pic/mydata_InDetBothClusterization_49}

\end{center}
\end{normalsize}
\end{center}
\end{frame}
%-------------------------------------------------
\begin{frame}
\frametitle{Times/event (seconds) for InDetSiTrackerSpacePointFinder}
\begin{center}
\begin{normalsize}
Full Detector (left)  and Inner Detectro(right) \\
\vspace*{-.2cm}
\begin{center}
\includegraphics[width=.40\textwidth,height=.40\textheight,type=png,ext=.png,read=.png]{/afs/cern.ch/user/a/atpathak/afswork/public/Pixel/ALLTXT_6Oct2017_r9876/pic/mydata_InDetSiTrackerSpacePointFinder_49}
\includegraphics[width=.40\textwidth,height=.40\textheight,type=png,ext=.png,read=.png]{/afs/cern.ch/user/a/atpathak/afswork/public/Pixel/ALLTXT_6Oct2017_r9876_ID/pic/mydata_InDetSiTrackerSpacePointFinder_49}

\end{center}
\end{normalsize}
\end{center}
\end{frame}
%-------------------------------------------------
\begin{frame}
\frametitle{Times/event (seconds) for InDetTrackParticles}
\begin{center}
\begin{normalsize}
Full Detector (left)  and Inner Detectro(right) \\
\vspace*{-.2cm}
\begin{center}
\includegraphics[width=.40\textwidth,height=.40\textheight,type=png,ext=.png,read=.png]{/afs/cern.ch/user/a/atpathak/afswork/public/Pixel/ALLTXT_6Oct2017_r9876/pic/mydata_InDetTrackParticles_49}
\includegraphics[width=.40\textwidth,height=.40\textheight,type=png,ext=.png,read=.png]{/afs/cern.ch/user/a/atpathak/afswork/public/Pixel/ALLTXT_6Oct2017_r9876_ID/pic/mydata_InDetTrackParticles_49}

\end{center}
\end{normalsize}
\end{center}
\end{frame}
%-------------------------------------------------
\begin{frame}
\frametitle{Times/event (seconds) for InDetPriVxFinder}
\begin{center}
\begin{normalsize}
Full Detector (left)  and Inner Detectro(right) \\
\vspace*{-.2cm}
\begin{center}
\includegraphics[width=.40\textwidth,height=.40\textheight,type=png,ext=.png,read=.png]{/afs/cern.ch/user/a/atpathak/afswork/public/Pixel/ALLTXT_6Oct2017_r9876/pic/mydata_InDetPriVxFinder_49}
\includegraphics[width=.40\textwidth,height=.40\textheight,type=png,ext=.png,read=.png]{/afs/cern.ch/user/a/atpathak/afswork/public/Pixel/ALLTXT_6Oct2017_r9876_ID/pic/mydata_InDetPriVxFinder_49}

\end{center}
\end{normalsize}
\end{center}
\end{frame}
%-------------------------------------------------
\begin{frame}
\frametitle{Times/event (seconds) for InDetVxLinkSetter}
\begin{center}
\begin{normalsize}
Full Detector (left)  and Inner Detectro(right) \\
\vspace*{-.2cm}
\begin{center}
\includegraphics[width=.40\textwidth,height=.40\textheight,type=png,ext=.png,read=.png]{/afs/cern.ch/user/a/atpathak/afswork/public/Pixel/ALLTXT_6Oct2017_r9876/pic/mydata_InDetVxLinkSetter_49}
\includegraphics[width=.40\textwidth,height=.40\textheight,type=png,ext=.png,read=.png]{/afs/cern.ch/user/a/atpathak/afswork/public/Pixel/ALLTXT_6Oct2017_r9876_ID/pic/mydata_InDetVxLinkSetter_49}

\end{center}
\end{normalsize}
\end{center}
\end{frame}
%-------------------------------------------------
\begin{frame}
\frametitle{Times/event (seconds) for InDetRecStatistics}
\begin{center}
\begin{normalsize}
Full Detector (left)  and Inner Detectro(right) \\
\vspace*{-.2cm}
\begin{center}
\includegraphics[width=.40\textwidth,height=.40\textheight,type=png,ext=.png,read=.png]{/afs/cern.ch/user/a/atpathak/afswork/public/Pixel/ALLTXT_6Oct2017_r9876/pic/mydata_InDetRecStatistics_49}
\includegraphics[width=.40\textwidth,height=.40\textheight,type=png,ext=.png,read=.png]{/afs/cern.ch/user/a/atpathak/afswork/public/Pixel/ALLTXT_6Oct2017_r9876_ID/pic/mydata_InDetRecStatistics_49}

\end{center}
\end{normalsize}
\end{center}
\end{frame}
%-------------------------------------------------
\begin{frame}
\frametitle{Times/event (seconds) for InDetCaloClusterROISelector}
\begin{center}
\begin{normalsize}
Full Detector (left)  and Inner Detectro(right) \\
\vspace*{-.2cm}
\begin{center}
\includegraphics[width=.40\textwidth,height=.40\textheight,type=png,ext=.png,read=.png]{/afs/cern.ch/user/a/atpathak/afswork/public/Pixel/ALLTXT_6Oct2017_r9876/pic/mydata_InDetCaloClusterROISelector_49}
%\includegraphics[width=.40\textwidth,height=.40\textheight,type=png,ext=.png,read=.png]{/afs/cern.ch/user/a/atpathak/afswork/public/Pixel/ALLTXT_6Oct2017_r9876_ID/pic/mydata_InDetCaloClusterROISelector_49}

\end{center}
\end{normalsize}
\end{center}
\end{frame}
%------------------------------------------------
\begin{frame}
\begin{Large}
\centerline{Reconstruction times for ITk studies r9877}
\end{Large}
\end{frame}
%-------------------------------------------------
\begin{frame}
\frametitle{Times/event (seconds) for AthAlgSeq}
\begin{center}
\begin{normalsize}
Full Detector (left)  and Inner Detectro(right) \\
\vspace*{-.2cm}
\begin{center}
\includegraphics[width=.40\textwidth,height=.40\textheight,type=png,ext=.png,read=.png]{/afs/cern.ch/user/a/atpathak/afswork/public/Pixel/ALLTXT_6Oct2017_r9877/pic/mydata_AthAlgSeq_49}
\includegraphics[width=.40\textwidth,height=.40\textheight,type=png,ext=.png,read=.png]{/afs/cern.ch/user/a/atpathak/afswork/public/Pixel/ALLTXT_6Oct2017_r9877_ID/pic/mydata_AthAlgSeq_49}

\end{center}
\end{normalsize}
\end{center}
\end{frame}
%-------------------------------------------------
\begin{frame}
\frametitle{Times/event (seconds) for InDetSiSpTrackFinderSLHC}
\begin{center}
\begin{normalsize}
Full Detector (left)  and Inner Detectro(right) \\
\vspace*{-.2cm}
\begin{center}
\includegraphics[width=.40\textwidth,height=.40\textheight,type=png,ext=.png,read=.png]{/afs/cern.ch/user/a/atpathak/afswork/public/Pixel/ALLTXT_6Oct2017_r9877/pic/mydata_InDetSiSpTrackFinderSLHC_49}
\includegraphics[width=.40\textwidth,height=.40\textheight,type=png,ext=.png,read=.png]{/afs/cern.ch/user/a/atpathak/afswork/public/Pixel/ALLTXT_6Oct2017_r9877_ID/pic/mydata_InDetSiSpTrackFinderSLHC_49}

\end{center}
\end{normalsize}
\end{center}
\end{frame}
%-------------------------------------------------
\begin{frame}
\frametitle{Times/event (seconds) for InDetAmbiguitySolverSLHC}
\begin{center}
\begin{normalsize}
Full Detector (left)  and Inner Detectro(right) \\
\vspace*{-.2cm}
\begin{center}
\includegraphics[width=.40\textwidth,height=.40\textheight,type=png,ext=.png,read=.png]{/afs/cern.ch/user/a/atpathak/afswork/public/Pixel/ALLTXT_6Oct2017_r9877/pic/mydata_InDetAmbiguitySolverSLHC_49}
\includegraphics[width=.40\textwidth,height=.40\textheight,type=png,ext=.png,read=.png]{/afs/cern.ch/user/a/atpathak/afswork/public/Pixel/ALLTXT_6Oct2017_r9877_ID/pic/mydata_InDetAmbiguitySolverSLHC_49}

\end{center}
\end{normalsize}
\end{center}
\end{frame}
%-------------------------------------------------
\begin{frame}
\frametitle{Times/event (seconds) for BothClusterization}
\begin{center}
\begin{normalsize}
Full Detector (left)  and Inner Detectro(right) \\
\vspace*{-.2cm}
\begin{center}
\includegraphics[width=.40\textwidth,height=.40\textheight,type=png,ext=.png,read=.png]{/afs/cern.ch/user/a/atpathak/afswork/public/Pixel/ALLTXT_6Oct2017_r9877/pic/mydata_InDetBothClusterization_49}
\includegraphics[width=.40\textwidth,height=.40\textheight,type=png,ext=.png,read=.png]{/afs/cern.ch/user/a/atpathak/afswork/public/Pixel/ALLTXT_6Oct2017_r9877_ID/pic/mydata_InDetBothClusterization_49}

\end{center}
\end{normalsize}
\end{center}
\end{frame}
%-------------------------------------------------
\begin{frame}
\frametitle{Times/event (seconds) for InDetSiTrackerSpacePointFinder}
\begin{center}
\begin{normalsize}
Full Detector (left)  and Inner Detectro(right) \\
\vspace*{-.2cm}
\begin{center}
\includegraphics[width=.40\textwidth,height=.40\textheight,type=png,ext=.png,read=.png]{/afs/cern.ch/user/a/atpathak/afswork/public/Pixel/ALLTXT_6Oct2017_r9877/pic/mydata_InDetSiTrackerSpacePointFinder_49}
\includegraphics[width=.40\textwidth,height=.40\textheight,type=png,ext=.png,read=.png]{/afs/cern.ch/user/a/atpathak/afswork/public/Pixel/ALLTXT_6Oct2017_r9877_ID/pic/mydata_InDetSiTrackerSpacePointFinder_49}

\end{center}
\end{normalsize}
\end{center}
\end{frame}
%-------------------------------------------------
\begin{frame}
\frametitle{Times/event (seconds) for InDetTrackParticles}
\begin{center}
\begin{normalsize}
Full Detector (left)  and Inner Detectro(right) \\
\vspace*{-.2cm}
\begin{center}
\includegraphics[width=.40\textwidth,height=.40\textheight,type=png,ext=.png,read=.png]{/afs/cern.ch/user/a/atpathak/afswork/public/Pixel/ALLTXT_6Oct2017_r9877/pic/mydata_InDetTrackParticles_49}
\includegraphics[width=.40\textwidth,height=.40\textheight,type=png,ext=.png,read=.png]{/afs/cern.ch/user/a/atpathak/afswork/public/Pixel/ALLTXT_6Oct2017_r9877_ID/pic/mydata_InDetTrackParticles_49}

\end{center}
\end{normalsize}
\end{center}
\end{frame}
%-------------------------------------------------
\begin{frame}
\frametitle{Times/event (seconds) for InDetPriVxFinder}
\begin{center}
\begin{normalsize}
Full Detector (left)  and Inner Detectro(right) \\
\vspace*{-.2cm}
\begin{center}
\includegraphics[width=.40\textwidth,height=.40\textheight,type=png,ext=.png,read=.png]{/afs/cern.ch/user/a/atpathak/afswork/public/Pixel/ALLTXT_6Oct2017_r9877/pic/mydata_InDetPriVxFinder_49}
\includegraphics[width=.40\textwidth,height=.40\textheight,type=png,ext=.png,read=.png]{/afs/cern.ch/user/a/atpathak/afswork/public/Pixel/ALLTXT_6Oct2017_r9877_ID/pic/mydata_InDetPriVxFinder_49}

\end{center}
\end{normalsize}
\end{center}
\end{frame}
%-------------------------------------------------
\begin{frame}
\frametitle{Times/event (seconds) for InDetVxLinkSetter}
\begin{center}
\begin{normalsize}
Full Detector (left)  and Inner Detectro(right) \\
\vspace*{-.2cm}
\begin{center}
\includegraphics[width=.40\textwidth,height=.40\textheight,type=png,ext=.png,read=.png]{/afs/cern.ch/user/a/atpathak/afswork/public/Pixel/ALLTXT_6Oct2017_r9877/pic/mydata_InDetVxLinkSetter_49}
\includegraphics[width=.40\textwidth,height=.40\textheight,type=png,ext=.png,read=.png]{/afs/cern.ch/user/a/atpathak/afswork/public/Pixel/ALLTXT_6Oct2017_r9877_ID/pic/mydata_InDetVxLinkSetter_49}

\end{center}
\end{normalsize}
\end{center}
\end{frame}
%-------------------------------------------------
\begin{frame}
\frametitle{Times/event (seconds) for InDetRecStatistics}
\begin{center}
\begin{normalsize}
Full Detector (left)  and Inner Detectro(right) \\
\vspace*{-.2cm}
\begin{center}
\includegraphics[width=.40\textwidth,height=.40\textheight,type=png,ext=.png,read=.png]{/afs/cern.ch/user/a/atpathak/afswork/public/Pixel/ALLTXT_6Oct2017_r9877/pic/mydata_InDetRecStatistics_49}
\includegraphics[width=.40\textwidth,height=.40\textheight,type=png,ext=.png,read=.png]{/afs/cern.ch/user/a/atpathak/afswork/public/Pixel/ALLTXT_6Oct2017_r9877_ID/pic/mydata_InDetRecStatistics_49}

\end{center}
\end{normalsize}
\end{center}
\end{frame}
%-------------------------------------------------
\begin{frame}
\frametitle{Times/event (seconds) for InDetCaloClusterROISelector}
\begin{center}
\begin{normalsize}
Full Detector (left)  and Inner Detectro(right) \\
\vspace*{-.2cm}
\begin{center}
\includegraphics[width=.40\textwidth,height=.40\textheight,type=png,ext=.png,read=.png]{/afs/cern.ch/user/a/atpathak/afswork/public/Pixel/ALLTXT_6Oct2017_r9877/pic/mydata_InDetCaloClusterROISelector_49}
%\includegraphics[width=.40\textwidth,height=.40\textheight,type=png,ext=.png,read=.png]{/afs/cern.ch/user/a/atpathak/afswork/public/Pixel/ALLTXT_6Oct2017_r9877_ID/pic/mydata_InDetCaloClusterROISelector_49}

\end{center}
\end{normalsize}
\end{center}
\end{frame}
%----------------------
\end{document}
%------------------------------------------------
\begin{frame}
\frametitle{Introduction}
\vspace*{0.0cm}
\begin{center}
\begin{itemize}
\item This is concluding part of my authorship qualification task.
\item Reconstruction times should preferably be benchmarked in a 
 controlled environment and not in a shared cluster like LXPLUS. For 
 this study, the reconstruction jobs were run  as sole user in time
slots granted via dedicated run time requests.
\item The cluster is Intel Xeon 7210, 1.3GHz 64-core CPU, 
  116GB RAM, 200GB SSD disk, 5.5TB raid array (CentOS 7.3.1611).
\item The jobs use $\sim$ 4 to 5 GB RAM. Up to 20 jobs were run in
  parallel. Each job was configured to run on a single processor. 
%PBS-like batch system has been installed 
%this week to automatize this configuration. But for the results
%presented here, jobs were run interactively via nohup. Batch system 
%will make future study easier.
\item Thanks to Dr.~Dang for installing ATLAS software using {\tt cvmfs} 
      running with squid cache in our new clusters at UofL.
\item {\bf HEP-SPEC06 score is 3.50 $\pm$ 0.15} (single core test),
      averaged over 5 random sets of cores. For 32 (64) core tests, scores are 3.64 (3.34)/core,
      consistent with single core tests.
\end{itemize}
\end{center}
\end{frame}
%------------------------------------------------
\begin{frame}
\frametitle{Introduction}
\vspace*{0.2cm}
\begin{itemize}
%\item This work is part of my authorship qualification task, which is
  %work in progress
%\item Reconstruction times should prefereably be benchmarked in
  %a controlled environment, eg. uniform cpuEfficiency/job
%\item Many thanks to Dr.~Dang for installing ATLAS software using {\tt
  %  cvmfs} in the new computing clusters at UofL.
\item 1000 $t\bar{t}$ events are studied for 4 pile-up in ITk and 1 in Run2
\item RAWtoESD output of 20 jobs run over 50 events are studied. % from 7 sets of jobs. Each set 
 %consists of 30 jobs run over (1,2,5,10,20,50,100) events, respectively.
\item The following 4 pile-up configurations for ITk are studied:
\end{itemize}
%\vspace*{.5cm}
%\caption{Configuration}
%\label{tab:config}
\begin{center}
\begin{tabular}{llll}\hline\hline
Pile-up & Reco release & Reco flag & Input RDO  \\\hline
$\langle \mu \rangle$ = 0    & 20.20.10.7    & AMI=r9877 & r-tag=9877 \\
$\langle \mu \rangle$ = 60 & 20.20.10.7    & AMI=r9876 & r-tag=9876 \\
$\langle \mu \rangle$ = 140         & 20.20.10.7    & AMI=r9875 & r-tag=9875 \\
$\langle \mu \rangle$ = 200         & 20.20.10.7    & AMI=r9871 & r-tag=9871 \\
\hline \hline
\end{tabular}

\vspace*{.1cm}
{\scriptsize{Samples: \url{mc15_14TeV.117050.PowhegPythia_P2011C_ttbar.recon.RDO.e2176_s3185_s3186_r9877/6/5/1}}}
\vspace*{.1cm}
\begin{itemize}
\item The following 1 pile-up configurations for Run2 is studied:
\end{itemize}
\vspace*{.2cm}
\begin{tabular}{llll}\hline\hline
Pile-up & Reco release & Reco flag & Input RDO  \\\hline
$\langle \mu \rangle$ = 20    & 20.7.5.1    & AMI=r7725 & r-tag=7725 \\
\hline \hline
\end{tabular}

\vspace*{.1cm}
{\scriptsize{Samples: \url{mc15_13TeV.410000.PowhegPythiaEvtGen_P2012_ttbar_hdamp172p5_nonallhad.recon.RDO.e3698_s2608_s2183_r7725}}}
\vspace*{.1cm}
\end{center}



%\vspace*{-.35cm}
%\begin{center}
%All configurations use for seeding:
%SiCombinatorialTrackFinderTool\_xk-01-00-34-06,
%SiSpacePointsSeedTool\_xk-01-00-35-05, 
%SiSPSeededTrackFinder-01-00-07
%\end{center}

\end{frame}
%------------------------------------------------
\begin{frame}
\begin{Large}
\centerline{Reconstruction times for Run 2 studies}
\end{Large}
\end{frame}
%-------------------------------------------------
\begin{frame}
\frametitle{Times/event (seconds) for AthAlgSeq (release 20.7.5.1)}

\begin{center}
\begin{normalsize}

%\begin{itemize}

%\item Time (minute) / event 
%\flushleft{} \\
%\item Timing for r9060 [left], r9133 [center] and r9139 [right]\\
%\item \\
%\item \\
\vspace*{.2cm}
\begin{center}
%\item For r9060(left) and For r9133(right) \\
\includegraphics[width=.40\textwidth,height=.40\textheight,type=png,ext=.png,read=.png]{/afs/cern.ch/user/a/atpathak/afswork/public/Pixel/ALLTXT_25Sep2017_r7725_1/pic/mydata_AthAlgSeq_50} 
\includegraphics[width=.55\textwidth,height=.40\textheight,type=pdf,ext=.pdf,read=.pdf]{/afs/cern.ch/user/s/swaban/afswork/public/Pixel/ALLTXT_19Sep2017_r9871/Slides/ATLASTPG_Reco_20170509_pg4}
\end{center}
%\item \\
%\end{itemize}
\vspace*{.3cm}
Our mean fitted time shown in left plot: 69.13 sec, 69.13*3.5 = 241.955 HS06 sec compares well with 20*12.5 = 250 sec from right plot by A. Lomosani (9 May 2017).
(\url{https://indico.cern.ch/event/638331/contributions/2586818/attachments/1456568/2247978/ATLASTPG_Reco_20170509.pdf}).
 {\textcolor{red} {Updated after private email communication with Antonio Limosani : aibuild010.cern.ch (the PMB benchmark machine) has 150.74/12 cores =12.5 HEPSPEC per core}}
\end{normalsize}
\end{center}
\end{frame}
%-------------------------------------------------
\begin{frame}
\frametitle{Times/event (seconds) for different algorithms}

\begin{center}
\begin{normalsize}

%\begin{itemize}

%\item Time (minute) / event 
InDetPixelClusterization (left)  and InDetSCTClusterization(right) \\
%\item Timing for r9060 [left], r9133 [center] and r9139 [right]\\
%\item \\
%\item \\
\vspace*{-.2cm}
\begin{center}
%\item For r9060(left) and For r9133(right) \\
%\includegraphics[width=.40\textwidth,height=.40\textheight,type=png,ext=.png,read=.png]{/afs/cern.ch/user/a/atpathak/afswork/public/Pixel/ALLTXT_25Sep2017_r7725_1/pic/mydata_AthAlgSeq_50} 
\includegraphics[width=.40\textwidth,height=.40\textheight,type=png,ext=.png,read=.png]{/afs/cern.ch/user/a/atpathak/afswork/public/Pixel/ALLTXT_25Sep2017_r7725_1/pic/mydata_InDetPixelClusterization_50}
\includegraphics[width=.40\textwidth,height=.40\textheight,type=png,ext=.png,read=.png]{/afs/cern.ch/user/a/atpathak/afswork/public/Pixel/ALLTXT_25Sep2017_r7725_1/pic/mydata_InDetSCTClusterization_50}
\\
\end{center}
%\item \\
%\end{itemize}
\end{normalsize}
\end{center}
\end{frame}

%------------------------------------------------
\begin{frame}
\begin{Large}
\centerline{Reconstruction times for ITk studies}
\end{Large}
\end{frame}
%-------------------------------------------------
\begin{frame}
\frametitle{Times/event (seconds) for AthAlgSeq}

\begin{center}
\begin{normalsize}

%\begin{itemize}

%\item Time (minute) / event 
for $\langle \mu \rangle$ =0 (left) and 60 (right) \\
%\item Timing for r9060 [left], r9133 [center] and r9139 [right]\\
%\item \\
%\item \\
\vspace*{-.2cm}
\begin{center}
%\item For r9060(left) and For r9133(right) \\
\includegraphics[width=.40\textwidth,height=.40\textheight,type=png,ext=.png,read=.png]{/afs/cern.ch/user/a/atpathak/afswork/public/Pixel/ALLTXT_19Sep2017_r9877/pic/mydata_AthAlgSeq_50} 
\includegraphics[width=.40\textwidth,height=.40\textheight,type=png,ext=.png,read=.png]{/afs/cern.ch/user/a/atpathak/afswork/public/Pixel/ALLTXT_19Sep2017_r9876/pic/mydata_AthAlgSeq_50}

%\item Time (minute) / event 

for $\langle \mu \rangle$ = 140 (left) and 200 (right) \\
\includegraphics[width=.40\textwidth,height=.40\textheight,type=png,ext=.png,read=.png]{/afs/cern.ch/user/a/atpathak/afswork/public/Pixel/ALLTXT_19Sep2017_r9875/pic/mydata_AthAlgSeq_50}
\includegraphics[width=.40\textwidth,height=.40\textheight,type=png,ext=.png,read=.png]{/afs/cern.ch/user/a/atpathak/afswork/public/Pixel/ALLTXT_19Sep2017_r9871/pic/mydata_AthAlgSeq_50}


\end{center}
%\item \\
%\end{itemize}
\end{normalsize}
\end{center}
\end{frame}
%-------------------------------------------------
\begin{frame}
\frametitle{Times/event (seconds) for InDetSiSpTrackFinderSLHC}

\begin{center}
\begin{normalsize}

%\begin{itemize}

%\item Time (minute) / event 
for $\langle \mu \rangle$ = 0 (left) and 60 (right) \\
%\item Timing for r9060 [left], r9133 [center] and r9139 [right]\\
%\item \\
%\item \\
\vspace*{-.2cm}
\begin{center}
%\item For r9060(left) and For r9133(right) \\
\includegraphics[width=.40\textwidth,height=.40\textheight,type=png,ext=.png,read=.png]{/afs/cern.ch/user/a/atpathak/afswork/public/Pixel/ALLTXT_19Sep2017_r9877/pic/mydata_InDetSiSpTrackFinderSLHC_50} 
\includegraphics[width=.40\textwidth,height=.40\textheight,type=png,ext=.png,read=.png]{/afs/cern.ch/user/a/atpathak/afswork/public/Pixel/ALLTXT_19Sep2017_r9876/pic/mydata_InDetSiSpTrackFinderSLHC_50}

%\item Time (minute) / event 

for $\langle \mu \rangle$ = 140 (left) and 200 (right) \\
\includegraphics[width=.40\textwidth,height=.40\textheight,type=png,ext=.png,read=.png]{/afs/cern.ch/user/a/atpathak/afswork/public/Pixel/ALLTXT_19Sep2017_r9875/pic/mydata_InDetSiSpTrackFinderSLHC_50}
\includegraphics[width=.40\textwidth,height=.40\textheight,type=png,ext=.png,read=.png]{/afs/cern.ch/user/a/atpathak/afswork/public/Pixel/ALLTXT_19Sep2017_r9871/pic/mydata_InDetSiSpTrackFinderSLHC_50}


\end{center}
%\item \\
%\end{itemize}
\end{normalsize}
\end{center}
\end{frame}
%-------------------------------------------------
\begin{frame}
\frametitle{Times/event (seconds) for InDetAmbiguitySolverSLHC}

\begin{center}
\begin{normalsize}

%\begin{itemize}

%\item Time (minute) / event 
for $\langle \mu \rangle$ = 0 (left) and 60 (right) \\
%\item Timing for r9060 [left], r9133 [center] and r9139 [right]\\
%\item \\
%\item \\
\vspace*{-.2cm}
\begin{center}
%\item For r9060(left) and For r9133(right) \\
\includegraphics[width=.40\textwidth,height=.40\textheight,type=png,ext=.png,read=.png]{/afs/cern.ch/user/a/atpathak/afswork/public/Pixel/ALLTXT_19Sep2017_r9877/pic/mydata_InDetAmbiguitySolverSLHC_50} 
\includegraphics[width=.40\textwidth,height=.40\textheight,type=png,ext=.png,read=.png]{/afs/cern.ch/user/a/atpathak/afswork/public/Pixel/ALLTXT_19Sep2017_r9876/pic/mydata_InDetAmbiguitySolverSLHC_50}

%\item Time (minute) / event 

for $\langle \mu \rangle$ = 140 (left) and 200 (right) \\
\includegraphics[width=.40\textwidth,height=.40\textheight,type=png,ext=.png,read=.png]{/afs/cern.ch/user/a/atpathak/afswork/public/Pixel/ALLTXT_19Sep2017_r9875/pic/mydata_InDetAmbiguitySolverSLHC_50}
\includegraphics[width=.40\textwidth,height=.40\textheight,type=png,ext=.png,read=.png]{/afs/cern.ch/user/a/atpathak/afswork/public/Pixel/ALLTXT_19Sep2017_r9871/pic/mydata_InDetAmbiguitySolverSLHC_50}


\end{center}
%\item \\
%\end{itemize}
\end{normalsize}
\end{center}
\end{frame}
%-------------------------------------------------
\begin{frame}
\frametitle{Times/event (seconds) for InDetPixelClusterization}

\begin{center}
\begin{normalsize}

%\begin{itemize}

%\item Time (minute) / event 
for $\langle \mu \rangle$ = 0 (left) and 60 (right) \\
%\item Timing for r9060 [left], r9133 [center] and r9139 [right]\\
%\item \\
%\item \\
\vspace*{-.2cm}
\begin{center}
%\item For r9060(left) and For r9133(right) \\
\includegraphics[width=.40\textwidth,height=.40\textheight,type=png,ext=.png,read=.png]{/afs/cern.ch/user/a/atpathak/afswork/public/Pixel/ALLTXT_19Sep2017_r9877/pic/mydata_InDetPixelClusterization_50} 
\includegraphics[width=.40\textwidth,height=.40\textheight,type=png,ext=.png,read=.png]{/afs/cern.ch/user/a/atpathak/afswork/public/Pixel/ALLTXT_19Sep2017_r9876/pic/mydata_InDetPixelClusterization_50}

%\item Time (minute) / event 

for $\langle \mu \rangle$ = 140 (left) and 200 (right) \\
\includegraphics[width=.40\textwidth,height=.40\textheight,type=png,ext=.png,read=.png]{/afs/cern.ch/user/a/atpathak/afswork/public/Pixel/ALLTXT_19Sep2017_r9875/pic/mydata_InDetPixelClusterization_50}
\includegraphics[width=.40\textwidth,height=.40\textheight,type=png,ext=.png,read=.png]{/afs/cern.ch/user/a/atpathak/afswork/public/Pixel/ALLTXT_19Sep2017_r9871/pic/mydata_InDetPixelClusterization_50}


\end{center}
%\item \\
%\end{itemize}
\end{normalsize}
\end{center}
\end{frame}

%-------------------------------------------------
\begin{frame}
\frametitle{Times/event (seconds) for InDetSCTClusterization}

\begin{center}
\begin{normalsize}

%\begin{itemize}

%\item Time (minute) / event 
for $\langle \mu \rangle$ = 0 (left) and 60 (right) \\
%\item Timing for r9060 [left], r9133 [center] and r9139 [right]\\
%\item \\
%\item \\
\vspace*{-.2cm}
\begin{center}
%\item For r9060(left) and For r9133(right) \\
\includegraphics[width=.40\textwidth,height=.40\textheight,type=png,ext=.png,read=.png]{/afs/cern.ch/user/a/atpathak/afswork/public/Pixel/ALLTXT_19Sep2017_r9877/pic/mydata_InDetSCTClusterization_50} 
\includegraphics[width=.40\textwidth,height=.40\textheight,type=png,ext=.png,read=.png]{/afs/cern.ch/user/a/atpathak/afswork/public/Pixel/ALLTXT_19Sep2017_r9876/pic/mydata_InDetSCTClusterization_50}

%\item Time (minute) / event 

for $\langle \mu \rangle$ = 140 (left) and 200 (right) \\
\includegraphics[width=.40\textwidth,height=.40\textheight,type=png,ext=.png,read=.png]{/afs/cern.ch/user/a/atpathak/afswork/public/Pixel/ALLTXT_19Sep2017_r9875/pic/mydata_InDetSCTClusterization_50}
\includegraphics[width=.40\textwidth,height=.40\textheight,type=png,ext=.png,read=.png]{/afs/cern.ch/user/a/atpathak/afswork/public/Pixel/ALLTXT_19Sep2017_r9871/pic/mydata_InDetSCTClusterization_50}


\end{center}
%\item \\
%\end{itemize}
\end{normalsize}
\end{center}
\end{frame}
%-------------------------------------------------
\begin{frame}
\frametitle{Benchmarking according to HEPSPEC06}

\begin{center}
\begin{normalsize}

%\begin{itemize}

%\item Time (minute) / event 
%\item Timing for r9060 [left], r9133 [center] and r9139 [right]\\
%\item \\
%\item \\
\vspace*{-.2cm}
\begin{center}
%\item For r9060(left) and For r9133(right) \\
\includegraphics[width=.65\textwidth,height=.95\textheight,type=pdf,ext=.pdf,read=.pdf]{/afs/cern.ch/user/a/atpathak/afswork/public/Pixel/ALLTXT_19Sep2017_r9871/All_PileUp_log}
\end{center}
%\item \\
%\end{itemize}
\end{normalsize}
\end{center}
\end{frame}
%------------------------------------------------
\begin{frame}
\frametitle{Summary of results (in HS06 seconds)}
\begin{itemize}
\item ITk
\end{itemize}
\vspace*{-0.5cm} 
\begin{table}

{\scalebox{.65}{
\begin{tabular}{l|cccc}\hline\hline
                           & \multicolumn {4}{c} {$\langle \mu \rangle$} \\
\cline{2-5}         & 0                & 60               & 140              & 200               \\ \hline
AthAlgSeq & 44.97 $\pm$ 0.56 & 249.04 $\pm$ 3.21 & 721.87 $\pm$ 9.84 & 1073.10 $\pm$ 8.00 \\ \hline
InDetAmbiguitySolverSLHC & 8.79 $\pm$ 0.11 & 77.17 $\pm$ 0.69 & 180.99 $\pm$ 1.39 & 261.98 $\pm$ 1.74 \\ \hline
InDetSiSpTrackFinderSLHC & 0.66 $\pm$ 0.01 & 10.94 $\pm$ 0.14 & 52.71 $\pm$ 0.63 & 119.96 $\pm$ 1.48 \\ \hline
InDetPixelClusterization & 0.32 $\pm$ 0.00 & 5.51 $\pm$ 0.08 & 14.06 $\pm$ 0.13 & 21.68 $\pm$ 0.13 \\ \hline 
InDetSCTClusterization & 0.13 $\pm$ 0.00 & 1.27 $\pm$ 0.02 & 2.34 $\pm$ 0.05 & 3.01 $\pm$ 0.05 \\ \hline
\hline
\end{tabular}
}}
\end{table}

\begin{itemize}
\item Run2
\end{itemize}
\vspace*{0.2cm}
\begin{table}

{\scalebox{.80}{
\begin{tabular}{l|c}\hline\hline
                          &  $\langle \mu \rangle$ = 20 \\\hline
AthAlgSeq & 241.97 $\pm$ 4.81 \\ \hline
InDetPixelClusterization & 0.72 $\pm$ 0.01 \\ \hline 
InDetSCTClusterization & 0.64 $\pm$ 0.01 \\ \hline
\hline
\end{tabular}
}}

\end{table}

\vspace*{1cm}
\end{frame}
\end{document}
%-------------------------------------------------
\begin{frame}
\frametitle{Benchmarking according to HEPSPEC06}

\begin{center}
\begin{normalsize}

%\begin{itemize}

%\item Time (minute) / event 
%\item Timing for r9060 [left], r9133 [center] and r9139 [right]\\
%\item \\
%\item \\
\vspace*{-.2cm}
\begin{center}
%\item For r9060(left) and For r9133(right) \\
\includegraphics[width=.65\textwidth,height=.65\textheight,type=pdf,ext=.pdf,read=.pdf]{/afs/cern.ch/user/a/atpathak/afswork/public/Pixel/ALLTXT_19Sep2017_r9871/All_PileUp_linear}
\end{center}
%\item \\
%\end{itemize}
\end{normalsize}
\end{center}
\end{frame}

%------------------------------------------------
\begin{frame}
\frametitle{cpuEfficiency vs \# of events for Nominal (r9060) }
\vspace*{0.1cm}
%\begin{center}
\begin{normalsize}
\begin{itemize}
\begin{small}
\item cpuEfficiency is an attribute in the {\texttt{jobReport.json}} file.
  It refers to fraction of time spent {\it{processing} } each event 
  normalized to total time in {\it{accessing~I/O~and~processing}} the event.
\item Left plot shows scatter-plot. Middle plot shows profile. Right
  plot shows fit : log(cpuEfficiency) = $p_0$ + $p_1$  log(N) + $p_2 $  [log(N)]$^2$
\end{small}
\end{itemize}
%\item Time (minute) / event 
%For (left to right)\\
%\item Timing for r9060 [left], r9133 [center] and r9139 [right]\\
%\item \\
%\item \\
\vspace*{0.05cm}
%\begin{center}
%\item For r9060(left) and For r9133(right) \\
\includegraphics[width=.35\textwidth,height=.35\textheight,type=png,ext=.png,read=.png]{/afs/cern.ch/user/a/atpathak/afswork/public/Pixel/ALLTXT_22May2017_r9060/CpuEfficiency_r9060} 
\includegraphics[width=.35\textwidth,height=.35\textheight,type=png,ext=.png,read=.png]{/afs/cern.ch/user/a/atpathak/afswork/public/Pixel/ALLTXT_22May2017_r9060/CpuEfficiency_r9060_new}
\includegraphics[width=.35\textwidth,height=.35\textheight,type=png,ext=.png,read=.png]{/afs/cern.ch/user/a/atpathak/afswork/public/Pixel/ALLTXT_22May2017_r9060/CpuEfficiency_r9060_old_1}
%\end{center}
%\item \\
%\end{itemize}
\vspace*{-0.2cm}
\begin{small}
The rms (shown by blue error bars in middle and right plots) for 1 event-jobs is
2.5\%, for 10 event-jobs it is 1.2\% and for 50 event-jobs it is
0.2\%. This is a measure of the systematic variation over each
set of 30 jobs and validates the claim that {\bf{all jobs are run in
a well controlled environment with uniform cpuEfficiency per job}}.
For $\geq$ 20 events, cpuEfficiency of $\geq$ 90\% is reached.
\end{small}
\end{normalsize}
%\end{center}
\end{frame}
%------------------------------------------------
\begin{frame}
\frametitle{cpuEfficiency vs \# of events for Long Strips (r9133) }
\vspace*{0.1cm}
%\begin{center}
\begin{normalsize}
\begin{itemize}
\begin{small}
\item cpuEfficiency is an attribute in the {\texttt{jobReport.json}} file.
  It refers to fraction of time spent {\it{processing} } each event 
  normalized to total time in {\it{accessing~I/O~and~processing}} the event.
\item Left plot shows scatter-plot. Middle plot shows profile. Right
  plot shows fit : log(cpuEfficiency) = $p_0$ + $p_1$  log(N) + $p_2 $  [log(N)]$^2$  
\end{small}
\end{itemize}
%\item Time (minute) / event 
%For (left to right)\\
%\item Timing for r9060 [left], r9133 [center] and r9139 [right]\\
%\item \\
%\item \\
\vspace*{0.05cm}
%\begin{center}
%\item For r9060(left) and For r9133(right) \\
\includegraphics[width=.35\textwidth,height=.35\textheight,type=png,ext=.png,read=.png]{/afs/cern.ch/user/a/atpathak/afswork/public/Pixel/ALLTXT_22May2017_r9133/CpuEfficiency_r9133} 
\includegraphics[width=.35\textwidth,height=.35\textheight,type=png,ext=.png,read=.png]{/afs/cern.ch/user/a/atpathak/afswork/public/Pixel/ALLTXT_22May2017_r9133/CpuEfficiency_r9133_new}
\includegraphics[width=.35\textwidth,height=.35\textheight,type=png,ext=.png,read=.png]{/afs/cern.ch/user/a/atpathak/afswork/public/Pixel/ALLTXT_22May2017_r9133/CpuEfficiency_r9133_old_1}
%\end{center}
%\item \\
%\end{itemize}
\vspace*{-0.2cm}
\begin{small}
The rms (shown by blue error bars in middle and right plots) for 1 event-jobs is
3.4\%, for 10 event-jobs it is 1.3\% and for 50 event-jobs it is
0.2\%. This is a measure of the systematic variation over each
set of 30 jobs and validates the claim that {\bf{all jobs are run in
a well controlled environment with uniform cpuEfficiency per job}}.
For $\geq$ 20 events, cpuEfficiency of $\geq$ 90\% is reached.  
\end{small}
\end{normalsize}
%\end{center}
\end{frame}
%------------------------------------------------
\begin{frame}
\frametitle{cpuEfficiency vs \# of events for BCL (r9139) }
\vspace*{0.1cm}
%\begin{center}
\begin{normalsize}
\begin{itemize}
\begin{small}
\item cpuEfficiency is an attribute in the {\texttt{jobReport.json}} file.
  It refers to fraction of time spent {\it{processing} } each event 
  normalized to total time in {\it{accessing~I/O~and~processing}} the event.
\item Left plot shows scatter-plot. Middle plot shows profile. Right
  plot shows fit : log(cpuEfficiency) = $p_0$ + $p_1$  log(N) + $p_2 $  [log(N)]$^2$   
\end{small}
\end{itemize}
%\item Time (minute) / event 
%For (left to right)\\
%\item Timing for r9060 [left], r9133 [center] and r9139 [right]\\
%\item \\
%\item \\
\vspace*{0.05cm}
%\begin{center}
%\item For r9060(left) and For r9133(right) \\
\includegraphics[width=.35\textwidth,height=.35\textheight,type=png,ext=.png,read=.png]{/afs/cern.ch/user/a/atpathak/afswork/public/Pixel/ALLTXT_22May2017_r9139/CpuEfficiency_r9139} 
\includegraphics[width=.35\textwidth,height=.35\textheight,type=png,ext=.png,read=.png]{/afs/cern.ch/user/a/atpathak/afswork/public/Pixel/ALLTXT_22May2017_r9139/CpuEfficiency_r9139_new}
\includegraphics[width=.35\textwidth,height=.35\textheight,type=png,ext=.png,read=.png]{/afs/cern.ch/user/a/atpathak/afswork/public/Pixel/ALLTXT_22May2017_r9139/CpuEfficiency_r9139_old_1}
%\end{center}
%\item \\
%\end{itemize}
\vspace*{-0.2cm}
\begin{small}
The rms (shown by blue error bars in middle and right plots) for 1 event-jobs is
2.2\%, for 10 event-jobs it is 1.2\% and for 50 event-jobs it is
0.3\%. This is a measure of the systematic variation over each
set of 30 jobs and validates the claim that {\bf{all jobs are run in
a well controlled environment with uniform cpuEfficiency per job}}.
For $\geq$ 20 events, cpuEfficiency of $\geq$ 90\% is reached.  
\end{small}
\end{normalsize}
%\end{center}
\end{frame}
%------------------------------------------------
\begin{frame}
\begin{huge}
\centerline{Reconstruction times for AthAlgSeq}
\end{huge}
\end{frame}
%-------------------------------------------------
\begin{frame}
\frametitle{Times (minutes) for AthAlgSeq for Nominal (r9060)}

\begin{center}
\begin{normalsize}

%\begin{itemize}

%\item Time (minute) / event 
for 30 sets of jobs, each over 1, 2, 5 events (left to right) \\
%\item Timing for r9060 [left], r9133 [center] and r9139 [right]\\
%\item \\
%\item \\
\vspace*{-.2cm}
\begin{center}
%\item For r9060(left) and For r9133(right) \\
\includegraphics[width=.22\textwidth,height=.3\textheight,type=png,ext=.png,read=.png]{/afs/cern.ch/user/a/atpathak/afswork/public/Pixel/ALLTXT_22May2017_r9060/pic/mydata_AthAlgSeq_1} 
\includegraphics[width=.22\textwidth,height=.3\textheight,type=png,ext=.png,read=.png]{/afs/cern.ch/user/a/atpathak/afswork/public/Pixel/ALLTXT_22May2017_r9060/pic/mydata_AthAlgSeq_2}
\includegraphics[width=.22\textwidth,height=.3\textheight,type=png,ext=.png,read=.png]{/afs/cern.ch/user/a/atpathak/afswork/public/Pixel/ALLTXT_22May2017_r9060/pic/mydata_AthAlgSeq_5} 
%\item Time (minute) / event 

for 30 sets of jobs, each over 10, 20, 50, 100  events (left to right) \\
\includegraphics[width=.22\textwidth,height=.3\textheight,type=png,ext=.png,read=.png]{/afs/cern.ch/user/a/atpathak/afswork/public/Pixel/ALLTXT_22May2017_r9060/pic/mydata_AthAlgSeq_10}
\includegraphics[width=.22\textwidth,height=.3\textheight,type=png,ext=.png,read=.png]{/afs/cern.ch/user/a/atpathak/afswork/public/Pixel/ALLTXT_22May2017_r9060/pic/mydata_AthAlgSeq_20}
\includegraphics[width=.22\textwidth,height=.3\textheight,type=png,ext=.png,read=.png]{/afs/cern.ch/user/a/atpathak/afswork/public/Pixel/ALLTXT_22May2017_r9060/pic/mydata_AthAlgSeq_50} 
\includegraphics[width=.22\textwidth,height=.3\textheight,type=png,ext=.png,read=.png]{/afs/cern.ch/user/a/atpathak/afswork/public/Pixel/ALLTXT_22May2017_r9060/pic/mydata_AthAlgSeq_100}  
\end{center}
%\item \\
%\end{itemize}
\end{normalsize}
\end{center}
\end{frame}
%------------------------------------------------
\begin{frame}
\frametitle{Times (minutes)  for AthAlgSeq for Nominal (r9060)}
\begin{normalsize}
\vspace*{-0.6cm}
\begin{itemize}
\begin{small}
\item The following plots show (fitted mean and fitted error ) / per event as function of NumberOfEvents
\item Fits to a constant value for [1 to 100], [2 to 100], [5 to 100]
  and [10 to 100] events, respectively, are shown below from left to right:
\end{small}
\end{itemize}
\vspace*{0.2cm}
\includegraphics[width=.27\textwidth,height=.3\textheight,type=png,ext=.png,read=.png]{/afs/cern.ch/user/a/atpathak/afswork/public/Pixel/ALLTXT_22May2017_r9060/AthAlgSeq_LinearFit_1} 
\includegraphics[width=.27\textwidth,height=.3\textheight,type=png,ext=.png,read=.png]{/afs/cern.ch/user/a/atpathak/afswork/public/Pixel/ALLTXT_22May2017_r9060/AthAlgSeq_LinearFit_2}
\includegraphics[width=.27\textwidth,height=.3\textheight,type=png,ext=.png,read=.png]{/afs/cern.ch/user/a/atpathak/afswork/public/Pixel/ALLTXT_22May2017_r9060/AthAlgSeq_LinearFit_5}
\includegraphics[width=.27\textwidth,height=.3\textheight,type=png,ext=.png,read=.png]{/afs/cern.ch/user/a/atpathak/afswork/public/Pixel/ALLTXT_22May2017_r9060/AthAlgSeq_LinearFit_10} 
\begin{itemize}
\begin{small}
\item Error (0.03583) from last fit $\sim$  Max - Min (= 0.011) of
  fitted means from the 4 sets of fits $\Rightarrow$ no additional 
  systematic error.
\end{small}
\end{itemize}

\end{normalsize}
\end{frame}
%--------------------------------------------------
\begin{frame}
\frametitle{AthAlgSeq times (minutes) for Nominal, LS and BCL}
\vspace*{1cm}
\begin{table}
\begin{center}
{\scalebox{.9}{
\begin{tabular}{lccc}\hline\hline
Fit $[{\mathrm{From}}, {\mathrm {To}}]$     & Nominal                     & Long strips           & BCL \\ \hline
$[1, 100]$                            & 6.339$\pm$0.03473 & 6.571$\pm$0.03696 & 6.319$\pm$0.03871\\
$[2, 100]$                            & 6.337$\pm$0.03499 & 6.564$\pm$0.0371  &  6.307$\pm$0.03916\\
$[5, 100]$                            & 6.333$\pm$0.03524 & 6.556$\pm$0.03739 & 6.296$\pm$0.03961\\
{\textcolor{red}{$[10, 100]$}} & {\textcolor{red} {6.328$\pm$ 0.03583} }&  {\textcolor{red} {6.548$\pm$0.03791} } & {\textcolor{red} {6.285$\pm$0.0409}} \\
\hline \hline
\end{tabular}
}}
\end{center}
\end{table}
\end{frame}
%------------------------------------------------
\begin{frame}
\begin{Large}
\centerline{Reconstruction times for InDetSiSpTrackFinderSLHC}
\end{Large}
\end{frame}
%------------------------------------------------
\begin{frame}
\frametitle{Times (min) for InDetSiSpTrackFinderSLHC Nominal}

\begin{center}
\begin{normalsize}

%\begin{itemize}

%\item Time (minute) / event 
for 30 sets of jobs, each over 1, 2, 5 events (left to right) \\
%\item Timing for r9060 [left], r9133 [center] and r9139 [right]\\
%\item \\
%\item \\
\vspace*{-.2cm}
\begin{center}
%\item For r9060(left) and For r9133(right) \\
\includegraphics[width=.22\textwidth,height=.3\textheight,type=png,ext=.png,read=.png]{/afs/cern.ch/user/a/atpathak/afswork/public/Pixel/ALLTXT_22May2017_r9060/pic/mydata_InDetSiSpTrackFinderSLHC_1} 
\includegraphics[width=.22\textwidth,height=.3\textheight,type=png,ext=.png,read=.png]{/afs/cern.ch/user/a/atpathak/afswork/public/Pixel/ALLTXT_22May2017_r9060/pic/mydata_InDetSiSpTrackFinderSLHC_2}
\includegraphics[width=.22\textwidth,height=.3\textheight,type=png,ext=.png,read=.png]{/afs/cern.ch/user/a/atpathak/afswork/public/Pixel/ALLTXT_22May2017_r9060/pic/mydata_InDetSiSpTrackFinderSLHC_5} 

%\item Time (minute) / event 

for 30 sets of jobs, each over 10, 20, 50, 100  events (left to right)\\
\includegraphics[width=.22\textwidth,height=.3\textheight,type=png,ext=.png,read=.png]{/afs/cern.ch/user/a/atpathak/afswork/public/Pixel/ALLTXT_22May2017_r9060/pic/mydata_InDetSiSpTrackFinderSLHC_10}
\includegraphics[width=.22\textwidth,height=.3\textheight,type=png,ext=.png,read=.png]{/afs/cern.ch/user/a/atpathak/afswork/public/Pixel/ALLTXT_22May2017_r9060/pic/mydata_InDetSiSpTrackFinderSLHC_20}
\includegraphics[width=.22\textwidth,height=.3\textheight,type=png,ext=.png,read=.png]{/afs/cern.ch/user/a/atpathak/afswork/public/Pixel/ALLTXT_22May2017_r9060/pic/mydata_InDetSiSpTrackFinderSLHC_50} 
\includegraphics[width=.22\textwidth,height=.3\textheight,type=png,ext=.png,read=.png]{/afs/cern.ch/user/a/atpathak/afswork/public/Pixel/ALLTXT_22May2017_r9060/pic/mydata_InDetSiSpTrackFinderSLHC_100}  
\end{center}
%\item \\
%\end{itemize}
\end{normalsize}
\end{center}
\end{frame}
%------------------------------------------------
\begin{frame}
\frametitle{Times (min) for InDetSiSpTrackFinderSLHC for Nominal}
\begin{normalsize}
\vspace*{-0.6cm}
\begin{itemize}
\begin{small}
\item The following plots show (fitted mean and fitted error ) / per event as function of NumberOfEvents
\item Fits to a constant value for [1 to 100], [2 to 100], [5 to 100]
  and [10 to 100] events, respectively, are shown below from left to right: 
\end{small}
\end{itemize}
\vspace*{0.2cm}
\includegraphics[width=.27\textwidth,height=.3\textheight,type=png,ext=.png,read=.png]{/afs/cern.ch/user/a/atpathak/afswork/public/Pixel/ALLTXT_22May2017_r9060/InDetSiSpTrackFinderSLHC_LinearFit_1} 
\includegraphics[width=.27\textwidth,height=.3\textheight,type=png,ext=.png,read=.png]{/afs/cern.ch/user/a/atpathak/afswork/public/Pixel/ALLTXT_22May2017_r9060/InDetSiSpTrackFinderSLHC_LinearFit_2}
\includegraphics[width=.27\textwidth,height=.3\textheight,type=png,ext=.png,read=.png]{/afs/cern.ch/user/a/atpathak/afswork/public/Pixel/ALLTXT_22May2017_r9060/InDetSiSpTrackFinderSLHC_LinearFit_5}
\includegraphics[width=.27\textwidth,height=.3\textheight,type=png,ext=.png,read=.png]{/afs/cern.ch/user/a/atpathak/afswork/public/Pixel/ALLTXT_22May2017_r9060/InDetSiSpTrackFinderSLHC_LinearFit_10} 
\begin{itemize}
\begin{small}
\item Error (0.005973) from last fit $\sim$ Max - Min (= 0.006) of
  fitted means from the 4 sets of fits $\Rightarrow$ no additional 
  systematic error.
\end{small}
\end{itemize}
\end{normalsize}
\end{frame}
%--------------------------------------------------
\begin{frame}
\frametitle{InDetSiSpTrackFinderSLHC times (min) for Nominal, LS and BCL}
\vspace*{1cm}
\begin{table}
\begin{center}
{\scalebox{.9}{
\begin{tabular}{lccc}\hline\hline
Fit $[{\mathrm{From}}, {\mathrm {To}}]$     & Nominal                     & Long strips           & BCL \\ \hline
$[1, 100]$                            & 1.181$\pm$0.005608 & 1.397$\pm$0.007079 & 1.167$\pm$0.004952\\
$[2, 100]$                            & 1.184$\pm$0.005686 & 1.396$\pm$0.007103  &  1.167$\pm$0.004978\\
$[5, 100]$                            & 1.185$\pm$0.005755 & 1.395$\pm$0.007151 & 1.166$\pm$0.00501\\
{\textcolor{red}{$[10, 100]$}} & {\textcolor{red} {1.187$\pm$ 0.005973} }&  {\textcolor{red} {1.398$\pm$0.007364} } & {\textcolor{red} {1.165$\pm$0.005104}} \\
\hline \hline
\end{tabular}
}}
\end{center}
\end{table}
\end{frame}
%------------------------------------------------
\begin{frame}
\begin{Large}
\centerline{Reconstruction times for InDetAmbiguitySolverSLHC}
\end{Large}
\end{frame}
%------------------------------------------------
\begin{frame}
\frametitle{Times (min) for InDetAmbiguitySolverSLHC for Nominal}

\begin{center}
\begin{normalsize}

%\begin{itemize}

%\item Time (minute) / event 
for 30 sets of jobs, each over 1, 2, 5 events (left to right) \\
%\item Timing for r9060 [left], r9133 [center] and r9139 [right]\\
%\item \\
%\item \\
\vspace*{-.2cm}
\begin{center}
%\item For r9060(left) and For r9133(right) \\
\includegraphics[width=.22\textwidth,height=.3\textheight,type=png,ext=.png,read=.png]{/afs/cern.ch/user/a/atpathak/afswork/public/Pixel/ALLTXT_22May2017_r9060/pic/mydata_InDetAmbiguitySolverSLHC_1} 
\includegraphics[width=.22\textwidth,height=.3\textheight,type=png,ext=.png,read=.png]{/afs/cern.ch/user/a/atpathak/afswork/public/Pixel/ALLTXT_22May2017_r9060/pic/mydata_InDetAmbiguitySolverSLHC_2}
\includegraphics[width=.22\textwidth,height=.3\textheight,type=png,ext=.png,read=.png]{/afs/cern.ch/user/a/atpathak/afswork/public/Pixel/ALLTXT_22May2017_r9060/pic/mydata_InDetAmbiguitySolverSLHC_5} 
%\item Time (minute) / event 

for 30 sets of jobs, each over 10, 20, 50, 100  events (left to right)\\
\includegraphics[width=.22\textwidth,height=.3\textheight,type=png,ext=.png,read=.png]{/afs/cern.ch/user/a/atpathak/afswork/public/Pixel/ALLTXT_22May2017_r9060/pic/mydata_InDetAmbiguitySolverSLHC_10}
\includegraphics[width=.22\textwidth,height=.3\textheight,type=png,ext=.png,read=.png]{/afs/cern.ch/user/a/atpathak/afswork/public/Pixel/ALLTXT_22May2017_r9060/pic/mydata_InDetAmbiguitySolverSLHC_20}
\includegraphics[width=.22\textwidth,height=.3\textheight,type=png,ext=.png,read=.png]{/afs/cern.ch/user/a/atpathak/afswork/public/Pixel/ALLTXT_22May2017_r9060/pic/mydata_InDetAmbiguitySolverSLHC_50} 
\includegraphics[width=.22\textwidth,height=.3\textheight,type=png,ext=.png,read=.png]{/afs/cern.ch/user/a/atpathak/afswork/public/Pixel/ALLTXT_22May2017_r9060/pic/mydata_InDetAmbiguitySolverSLHC_100}  
\end{center}
%\item \\
%\end{itemize}
\end{normalsize}
\end{center}
\end{frame}
%------------------------------------------------
\begin{frame}
\frametitle{Times (min) for InDetAmbiguitySolverSLHC for Nominal}
\begin{normalsize}
\vspace*{-0.6cm}
\begin{itemize}
\begin{small}
\item The following plots show (fitted mean and fitted error ) / per event as function of NumberOfEvents
\item Fits to a constant value for [1 to 100], [2 to 100], [5 to 100]
  and [10 to 100] events, respectively, are shown below from left to right: 
\end{small}
\end{itemize}
\vspace*{0.2cm}
\includegraphics[width=.27\textwidth,height=.3\textheight,type=png,ext=.png,read=.png]{/afs/cern.ch/user/a/atpathak/afswork/public/Pixel/ALLTXT_22May2017_r9060/InDetAmbiguitySolverSLHC_LinearFit_1} 
\includegraphics[width=.27\textwidth,height=.3\textheight,type=png,ext=.png,read=.png]{/afs/cern.ch/user/a/atpathak/afswork/public/Pixel/ALLTXT_22May2017_r9060/InDetAmbiguitySolverSLHC_LinearFit_2}
\includegraphics[width=.27\textwidth,height=.3\textheight,type=png,ext=.png,read=.png]{/afs/cern.ch/user/a/atpathak/afswork/public/Pixel/ALLTXT_22May2017_r9060/InDetAmbiguitySolverSLHC_LinearFit_5}
\includegraphics[width=.27\textwidth,height=.3\textheight,type=png,ext=.png,read=.png]{/afs/cern.ch/user/a/atpathak/afswork/public/Pixel/ALLTXT_22May2017_r9060/InDetAmbiguitySolverSLHC_LinearFit_10} 
\begin{itemize}
\begin{small}
\item Error (0.001345) from last fit $\sim$ Max - Min (= 0.002) of
  fitted means from the 4 sets of fits $\Rightarrow$ no additional 
  systematic error.
\end{small}
\end{itemize}
\end{normalsize}
\end{frame}
%--------------------------------------------------
\begin{frame}
\frametitle{InDetAmbiguitySolverSLHC times (min) for Nominal, LS and BCL}
\vspace*{1cm}
\begin{table}
\begin{center}
{\scalebox{.9}{
\begin{tabular}{lccc}\hline\hline
Fit $[{\mathrm{From}}, {\mathrm {To}}]$     & Nominal                     & Long strips           & BCL \\ \hline
$[1, 100]$                            & 0.3327$\pm$0.001166 & 0.3376$\pm$0.001077 & 0.3338$\pm$0.001071\\
$[2, 100]$                            & 0.3336$\pm$0.001238 & 0.3375$\pm$0.001085  &  0.3338$\pm$0.001085\\
$[5, 100]$                            & 0.334$\pm$0.001269 & 0.3377$\pm$0.001103 & 0.3337$\pm$0.001101\\
{\textcolor{red}{$[10, 100]$}} & {\textcolor{red} {0.3347$\pm$ 0.001345} }&  {\textcolor{red} {0.3379$\pm$0.001137} } & {\textcolor{red} {0.3339$\pm$0.001144}} \\
\hline \hline
\end{tabular}
}}
\end{center}
\end{table}
\end{frame}
%------------------------------------------------
\begin{frame}
\begin{Large}
\centerline{Reconstruction times for InDetSCT\_Clusterization}
\end{Large}
\end{frame}
%------------------------------------------------
\begin{frame}
\frametitle{Times (Sec) for InDetSCT\_Clusterization for Nominal}

\begin{center}
\begin{normalsize}

%\begin{itemize}

%\item Time (Second) / event 
for 30 sets of jobs, each over 1, 2, 5 events (left to right)\\
%\item Timing for r9060 [left], r9133 [center] and r9139 [right]\\
%\item \\
%\item \\
\vspace*{-.2cm}
\begin{center}
%\item For r9060(left) and For r9133(right) \\
\includegraphics[width=.22\textwidth,height=.3\textheight,type=png,ext=.png,read=.png]{/afs/cern.ch/user/a/atpathak/afswork/public/Pixel/ALLTXT_22May2017_r9060/pic/mydata_InDetSCTClusterization_1} 
\includegraphics[width=.22\textwidth,height=.3\textheight,type=png,ext=.png,read=.png]{/afs/cern.ch/user/a/atpathak/afswork/public/Pixel/ALLTXT_22May2017_r9060/pic/mydata_InDetSCTClusterization_2}
\includegraphics[width=.22\textwidth,height=.3\textheight,type=png,ext=.png,read=.png]{/afs/cern.ch/user/a/atpathak/afswork/public/Pixel/ALLTXT_22May2017_r9060/pic/mydata_InDetSCTClusterization_5} 

%\item Time (Second) / event 

for 30 sets of jobs, each over 10, 20, 50, 100  events (left to right) \\
\includegraphics[width=.22\textwidth,height=.3\textheight,type=png,ext=.png,read=.png]{/afs/cern.ch/user/a/atpathak/afswork/public/Pixel/ALLTXT_22May2017_r9060/pic/mydata_InDetSCTClusterization_10}
\includegraphics[width=.22\textwidth,height=.3\textheight,type=png,ext=.png,read=.png]{/afs/cern.ch/user/a/atpathak/afswork/public/Pixel/ALLTXT_22May2017_r9060/pic/mydata_InDetSCTClusterization_20}
\includegraphics[width=.22\textwidth,height=.3\textheight,type=png,ext=.png,read=.png]{/afs/cern.ch/user/a/atpathak/afswork/public/Pixel/ALLTXT_22May2017_r9060/pic/mydata_InDetSCTClusterization_50} 
\includegraphics[width=.22\textwidth,height=.3\textheight,type=png,ext=.png,read=.png]{/afs/cern.ch/user/a/atpathak/afswork/public/Pixel/ALLTXT_22May2017_r9060/pic/mydata_InDetSCTClusterization_100}  
\end{center}
%\item \\
\begin{small}
There is also a 2$^{nd}$ peak around 1 sec (from pathological events).
\end{small}
%\end{itemize}
\end{normalsize}
\end{center}
\end{frame}
%------------------------------------------------
\begin{frame}
\frametitle{Times (Sec) for InDetSCT\_Clusterization for Nominal}
\begin{normalsize}
\vspace*{-0.6cm}
\begin{itemize}
\begin{small}
\item The following plots show (fitted mean and fitted error ) / per event as function of NumberOfEvents
\item Fits to a constant value for [1 to 100], [2 to 100], [5 to 100]
  and [10 to 100] events, respectively, are shown below from left to right: 
\end{small}
\end{itemize}
\vspace*{0.2cm}
\includegraphics[width=.27\textwidth,height=.3\textheight,type=png,ext=.png,read=.png]{/afs/cern.ch/user/a/atpathak/afswork/public/Pixel/ALLTXT_22May2017_r9060/InDetSCTClusterization_LinearFit_1} 
\includegraphics[width=.27\textwidth,height=.3\textheight,type=png,ext=.png,read=.png]{/afs/cern.ch/user/a/atpathak/afswork/public/Pixel/ALLTXT_22May2017_r9060/InDetSCTClusterization_LinearFit_2}
\includegraphics[width=.27\textwidth,height=.3\textheight,type=png,ext=.png,read=.png]{/afs/cern.ch/user/a/atpathak/afswork/public/Pixel/ALLTXT_22May2017_r9060/InDetSCTClusterization_LinearFit_5}
\includegraphics[width=.27\textwidth,height=.3\textheight,type=png,ext=.png,read=.png]{/afs/cern.ch/user/a/atpathak/afswork/public/Pixel/ALLTXT_22May2017_r9060/InDetSCTClusterization_LinearFit_10} 
\begin{itemize}
\begin{small}
\item Error (0.001428) from last fit $\sim$  Max - Min (=0.0016 ) of
  fitted means from the 4 sets of fits $\Rightarrow$ no additional 
  systematic error.
\end{small}
\end{itemize}

\end{normalsize}
\end{frame}
%--------------------------------------------------
\begin{frame}
\frametitle{InDetSCT\_Clusterization times (Sec) for Nominal, LS and BCL}
\vspace*{1cm}
\begin{table}
\begin{center}
{\scalebox{.9}{
\begin{tabular}{lccc}\hline\hline
Fit $[{\mathrm{From}}, {\mathrm {To}}]$     & Nominal                     & Long strips           & BCL \\ \hline
$[1, 100]$                            & 0.8073$\pm$0.001411 & 0.8006$\pm$0.001209 & 0.8129$\pm$0.001399\\
$[2, 100]$                            & 0.8063$\pm$0.001417 & 0.8$\pm$0.001212  &  0.8117$\pm$0.001405\\
$[5, 100]$                            & 0.806$\pm$0.001421 & 0.7996$\pm$0.001216 & 0.8109$\pm$0.00141\\
{\textcolor{red}{$[10, 100]$}} & {\textcolor{red} {0.8057$\pm$ 0.001428} }&  {\textcolor{red} {0.7994$\pm$0.001216} } & {\textcolor{red} {0.8107$\pm$0.001418}} \\
\hline \hline
\end{tabular}
}}
\end{center}
\end{table}
\end{frame}
%------------------------------------------------
\begin{frame}
\frametitle{Summary of results}
\begin{table}
\begin{center}
{\scalebox{.69}{
\begin{tabular}{lccc}\hline\hline
Algorithm                        & Nominal  & Long strips & BCL  \\\hline
AthAlgSeq:Execute(min)  &  6.328$\pm$ 0.03583 & 6.548$\pm$0.03791 & 6.285$\pm$0.0409    \\ 
InDetSiSpTrackFinderSLHC:Execute(min) &1.187$\pm$0.005973 & 1.398$\pm$0.007364 & 1.165$\pm$0.005104  \\ 
InDetAmbiguitySolverSLHC:Execute(min) &0.3347$\pm$0.001345 & 0.3379$\pm$0.001137 & 0.3339$\pm$0.001144\\
InDetSCT\_Clusterization:Execute(sec)    & 0.8057$\pm$0.001428 & 0.7994$\pm$0.001216  & 0.8107$\pm$0.001418\\\hline \hline
\end{tabular}
}}
\end{center}
\end{table}

\vspace*{1cm}
\end{frame}
%------------------------------------------------
\begin{frame}
\frametitle{\begin{normalsize} Conclusion and Outlook \end{normalsize}}

%\begin{center}
\begin{normalsize}

\begin{itemize}
\begin{small}
\item Setup in place to benchmark performance in controlled
 environment. For jobs with $\geq$ 10 events, variations of 
performance in terms of cpuEfficiency are 
$\sim$ 1\%. For 50 events, variations are $\sim$ 0.2\%.
\item Studied reconstruction times for Nominal, Long Strip and BCL configurations\\
\item Looked at "Execute" times for 4 different algorithms : AthAlgSeq,
  InDetSiSpTrackFinderSLHC,  InDetAmbiguitySolverSLHC and
  InDetSCT\_Clusterization \\
\item Studied 7 sets of jobs , with 30 runs per job. 7 sets correspond
  to 1, 2, 5, 10, 20, 50 and 100 events. Stable results obtained for
  sets with $\geq$ 10 events. In future, may use more than 30 runs per set.
\item Batch system installed in our cluster: will make future studies easier.
\item Future plans are to study timing as function of pileup\\
\end{small}
\end{itemize}
\end{normalsize}
%\end{center}
\end{frame}
%------------------------------------------------
\begin{frame}
\begin{center}
\begin{Huge}
\centerline{Back up Slides}
\end{Huge}
\end{center}
\end{frame}
%------------------------------------------------
\begin{frame}
\begin{huge}
\centerline{Reconstruction times for AthAlgSeq}
\end{huge}
\end{frame}
%------------------------------------------------
\begin{frame}
\frametitle{Times (min) for AthAlgSeq for Long Strips}

\begin{center}
\begin{normalsize}

%\begin{itemize}

%\item Time (minute) / event 
for 30 sets of jobs, each over 1, 2, 5 events (left to right) \\
%\item Timing for r9060 [left], r9133 [center] and r9139 [right]\\
%\item \\
%\item \\
\vspace*{-.2cm}
\begin{center}
%\item For r9060(left) and For r9133(right) \\
\includegraphics[width=.22\textwidth,height=.3\textheight,type=png,ext=.png,read=.png]{/afs/cern.ch/user/a/atpathak/afswork/public/Pixel/ALLTXT_22May2017_r9133/pic/mydata_AthAlgSeq_1} 
\includegraphics[width=.22\textwidth,height=.3\textheight,type=png,ext=.png,read=.png]{/afs/cern.ch/user/a/atpathak/afswork/public/Pixel/ALLTXT_22May2017_r9133/pic/mydata_AthAlgSeq_2}
\includegraphics[width=.22\textwidth,height=.3\textheight,type=png,ext=.png,read=.png]{/afs/cern.ch/user/a/atpathak/afswork/public/Pixel/ALLTXT_22May2017_r9133/pic/mydata_AthAlgSeq_5} 
%\item Time (minute) / event 

for 30 sets of jobs, each over 10, 20, 50, 100 events(left to right) \\
\includegraphics[width=.22\textwidth,height=.3\textheight,type=png,ext=.png,read=.png]{/afs/cern.ch/user/a/atpathak/afswork/public/Pixel/ALLTXT_22May2017_r9133/pic/mydata_AthAlgSeq_10}
\includegraphics[width=.22\textwidth,height=.3\textheight,type=png,ext=.png,read=.png]{/afs/cern.ch/user/a/atpathak/afswork/public/Pixel/ALLTXT_22May2017_r9133/pic/mydata_AthAlgSeq_20}
\includegraphics[width=.22\textwidth,height=.3\textheight,type=png,ext=.png,read=.png]{/afs/cern.ch/user/a/atpathak/afswork/public/Pixel/ALLTXT_22May2017_r9133/pic/mydata_AthAlgSeq_50} 
\includegraphics[width=.22\textwidth,height=.3\textheight,type=png,ext=.png,read=.png]{/afs/cern.ch/user/a/atpathak/afswork/public/Pixel/ALLTXT_22May2017_r9133/pic/mydata_AthAlgSeq_100}  
\end{center}
%\item \\
%\end{itemize}
\end{normalsize}
\end{center}
\end{frame}
%------------------------------------------------
\begin{frame}
\frametitle{Times (minutes) for AthAlgSeq for Long Strips}
\begin{normalsize}
\vspace*{-0.6cm}
\begin{itemize}
\begin{small}
\item The following plots show (fitted mean and fitted error ) / per event as function of NumberOfEvents
\item Fits to a constant value for [1 to 100], [2 to 100], [5 to 100]
  and [10 to 100] events, respectively, are shown below from left to right:
 \end{small}
\end{itemize}
\vspace*{0.2cm}
\includegraphics[width=.27\textwidth,height=.3\textheight,type=png,ext=.png,read=.png]{/afs/cern.ch/user/a/atpathak/afswork/public/Pixel/ALLTXT_22May2017_r9133/AthAlgSeq_LinearFit_1} 
\includegraphics[width=.27\textwidth,height=.3\textheight,type=png,ext=.png,read=.png]{/afs/cern.ch/user/a/atpathak/afswork/public/Pixel/ALLTXT_22May2017_r9133/AthAlgSeq_LinearFit_2}
\includegraphics[width=.27\textwidth,height=.3\textheight,type=png,ext=.png,read=.png]{/afs/cern.ch/user/a/atpathak/afswork/public/Pixel/ALLTXT_22May2017_r9133/AthAlgSeq_LinearFit_5}
\includegraphics[width=.27\textwidth,height=.3\textheight,type=png,ext=.png,read=.png]{/afs/cern.ch/user/a/atpathak/afswork/public/Pixel/ALLTXT_22May2017_r9133/AthAlgSeq_LinearFit_10} 
\begin{itemize}
\begin{small}
\item Error (0.03791) from last fit $\sim$  Max - Min (= 0.023) of
  fitted means from the 4 sets of fits $\Rightarrow$ no additional 
  systematic error.
\end{small}
\end{itemize}
\end{normalsize}
\end{frame}
%------------------------------------------------
\begin{frame}
\frametitle{Times (minutes)  for AthAlgSeq for BCL (r9139)}

\begin{center}
\begin{normalsize}

%\begin{itemize}

%\item Time (minute) / event 
for 30 sets of jobs, each over 1,  2,  5  events (left to right) \\
%\item Timing for r9060 [left], r9133 [center] and r9139 [right]\\
%\item \\
%\item \\
\vspace*{-.2cm}
\begin{center}
%\item For r9060(left) and For r9133(right) \\
\includegraphics[width=.22\textwidth,height=.3\textheight,type=png,ext=.png,read=.png]{/afs/cern.ch/user/a/atpathak/afswork/public/Pixel/ALLTXT_22May2017_r9139/pic/mydata_AthAlgSeq_1} 
\includegraphics[width=.22\textwidth,height=.3\textheight,type=png,ext=.png,read=.png]{/afs/cern.ch/user/a/atpathak/afswork/public/Pixel/ALLTXT_22May2017_r9139/pic/mydata_AthAlgSeq_2}
\includegraphics[width=.22\textwidth,height=.3\textheight,type=png,ext=.png,read=.png]{/afs/cern.ch/user/a/atpathak/afswork/public/Pixel/ALLTXT_22May2017_r9139/pic/mydata_AthAlgSeq_5} 
%\item Time (minute) / event  

for 30 sets of jobs, each over 10,  20,  50,  100 events (left to right) \\
\includegraphics[width=.22\textwidth,height=.3\textheight,type=png,ext=.png,read=.png]{/afs/cern.ch/user/a/atpathak/afswork/public/Pixel/ALLTXT_22May2017_r9139/pic/mydata_AthAlgSeq_10}
\includegraphics[width=.22\textwidth,height=.3\textheight,type=png,ext=.png,read=.png]{/afs/cern.ch/user/a/atpathak/afswork/public/Pixel/ALLTXT_22May2017_r9139/pic/mydata_AthAlgSeq_20}
\includegraphics[width=.22\textwidth,height=.3\textheight,type=png,ext=.png,read=.png]{/afs/cern.ch/user/a/atpathak/afswork/public/Pixel/ALLTXT_22May2017_r9139/pic/mydata_AthAlgSeq_50} 
\includegraphics[width=.22\textwidth,height=.3\textheight,type=png,ext=.png,read=.png]{/afs/cern.ch/user/a/atpathak/afswork/public/Pixel/ALLTXT_22May2017_r9139/pic/mydata_AthAlgSeq_100}  
\end{center}
%\item \\
%\end{itemize}
\end{normalsize}
\end{center}
\end{frame}
%------------------------------------------------
\begin{frame}
\frametitle{Times (minutes) for AthAlgSeq for BCL (r9139)}
\begin{normalsize}
\vspace*{-0.6cm}
\begin{itemize}
\begin{small}
\item The following plots show (fitted mean and fitted error ) / per event as function of NumberOfEvents
\item Fits to a constant value for [1 to 100], [2 to 100], [5 to 100]
  and [10 to 100] events, respectively, are shown below from left to right:
\end{small}
\end{itemize}
\vspace*{0.2cm}
\includegraphics[width=.27\textwidth,height=.3\textheight,type=png,ext=.png,read=.png]{/afs/cern.ch/user/a/atpathak/afswork/public/Pixel/ALLTXT_22May2017_r9139/AthAlgSeq_LinearFit_1} 
\includegraphics[width=.27\textwidth,height=.3\textheight,type=png,ext=.png,read=.png]{/afs/cern.ch/user/a/atpathak/afswork/public/Pixel/ALLTXT_22May2017_r9139/AthAlgSeq_LinearFit_2}
\includegraphics[width=.27\textwidth,height=.3\textheight,type=png,ext=.png,read=.png]{/afs/cern.ch/user/a/atpathak/afswork/public/Pixel/ALLTXT_22May2017_r9139/AthAlgSeq_LinearFit_5}
\includegraphics[width=.27\textwidth,height=.3\textheight,type=png,ext=.png,read=.png]{/afs/cern.ch/user/a/atpathak/afswork/public/Pixel/ALLTXT_22May2017_r9139/AthAlgSeq_LinearFit_10} 
\begin{itemize}
\begin{small}
\item Error (0.0409) from last fit $\sim$  Max - Min (= 0.034) of
  fitted means from the 4 sets of fits $\Rightarrow$ no additional 
  systematic error.
\end{small}
\end{itemize}

\end{normalsize}
\end{frame}
%------------------------------------------------
\begin{frame}
\begin{Large}
\centerline{Reconstruction times for InDetSiSpTrackFinderSLHC}
\end{Large}
\end{frame}
%------------------------------------------------
\begin{frame}
\frametitle{Times (min) for InDetSiSpTrackFinderSLHC for Long Strips}

\begin{center}
\begin{normalsize}

%\begin{itemize}

%\item Time (minute) / event 
for 30 sets of jobs, each over 1, 2, 5 events (left to right) \\
%\item Timing for r9060 [left], r9133 [center] and r9139 [right]\\
%\item \\
%\item \\
\vspace*{-.2cm}
\begin{center}
%\item For r9060(left) and For r9133(right) \\
\includegraphics[width=.22\textwidth,height=.3\textheight,type=png,ext=.png,read=.png]{/afs/cern.ch/user/a/atpathak/afswork/public/Pixel/ALLTXT_22May2017_r9133/pic/mydata_InDetSiSpTrackFinderSLHC_1} 
\includegraphics[width=.22\textwidth,height=.3\textheight,type=png,ext=.png,read=.png]{/afs/cern.ch/user/a/atpathak/afswork/public/Pixel/ALLTXT_22May2017_r9133/pic/mydata_InDetSiSpTrackFinderSLHC_2}
\includegraphics[width=.22\textwidth,height=.3\textheight,type=png,ext=.png,read=.png]{/afs/cern.ch/user/a/atpathak/afswork/public/Pixel/ALLTXT_22May2017_r9133/pic/mydata_InDetSiSpTrackFinderSLHC_5} 
%\item Time (minute) / event 

for 30 sets of jobs, each over 10, 20, 50, 100  events (left to right)\\
\includegraphics[width=.22\textwidth,height=.3\textheight,type=png,ext=.png,read=.png]{/afs/cern.ch/user/a/atpathak/afswork/public/Pixel/ALLTXT_22May2017_r9133/pic/mydata_InDetSiSpTrackFinderSLHC_10}
\includegraphics[width=.22\textwidth,height=.3\textheight,type=png,ext=.png,read=.png]{/afs/cern.ch/user/a/atpathak/afswork/public/Pixel/ALLTXT_22May2017_r9133/pic/mydata_InDetSiSpTrackFinderSLHC_20}
\includegraphics[width=.22\textwidth,height=.3\textheight,type=png,ext=.png,read=.png]{/afs/cern.ch/user/a/atpathak/afswork/public/Pixel/ALLTXT_22May2017_r9133/pic/mydata_InDetSiSpTrackFinderSLHC_50} 
\includegraphics[width=.22\textwidth,height=.3\textheight,type=png,ext=.png,read=.png]{/afs/cern.ch/user/a/atpathak/afswork/public/Pixel/ALLTXT_22May2017_r9133/pic/mydata_InDetSiSpTrackFinderSLHC_100}  
\end{center}
%\item \\
%\end{itemize}
\end{normalsize}
\end{center}
\end{frame}
%------------------------------------------------
\begin{frame}
\frametitle{Times (min) for InDetSiSpTrackFinderSLHC for Long Strips}
\begin{normalsize}
\vspace*{-0.6cm}
\begin{itemize}
\begin{small}
\item The following plots show (fitted mean and fitted error ) / per event as function of NumberOfEvents
\item Fits to a constant value for [1 to 100], [2 to 100], [5 to 100]
  and [10 to 100] events, respectively, are shown below from left to right: 
\end{small}
\end{itemize}
\vspace*{0.2cm}
\includegraphics[width=.27\textwidth,height=.3\textheight,type=png,ext=.png,read=.png]{/afs/cern.ch/user/a/atpathak/afswork/public/Pixel/ALLTXT_22May2017_r9133/InDetSiSpTrackFinderSLHC_LinearFit_1} 
\includegraphics[width=.27\textwidth,height=.3\textheight,type=png,ext=.png,read=.png]{/afs/cern.ch/user/a/atpathak/afswork/public/Pixel/ALLTXT_22May2017_r9133/InDetSiSpTrackFinderSLHC_LinearFit_2}
\includegraphics[width=.27\textwidth,height=.3\textheight,type=png,ext=.png,read=.png]{/afs/cern.ch/user/a/atpathak/afswork/public/Pixel/ALLTXT_22May2017_r9133/InDetSiSpTrackFinderSLHC_LinearFit_5}
\includegraphics[width=.27\textwidth,height=.3\textheight,type=png,ext=.png,read=.png]{/afs/cern.ch/user/a/atpathak/afswork/public/Pixel/ALLTXT_22May2017_r9133/InDetSiSpTrackFinderSLHC_LinearFit_10} 
\begin{itemize}
\begin{small}
\item Error (0.007364) from last fit $\sim$ Max - Min (= 0.003) of
  fitted means from the 4 sets of fits $\Rightarrow$ no additional 
  systematic error.
\end{small}
\end{itemize}
\end{normalsize}
\end{frame}
%------------------------------------------------
\begin{frame}
\frametitle{Times (min) forInDetSiSpTrackFinderSLHC for BCL (r9139)}

\begin{center}
\begin{normalsize}

%\begin{itemize}

%\item Time (minute) / event 
for 30 sets of jobs, each over 1, 2, 5 events (left to right) \\
%\item Timing for r9060 [left], r9133 [center] and r9139 [right]\\
%\item \\
%\item \\
\vspace*{-.2cm}
\begin{center}
%\item For r9060(left) and For r9133(right) \\
\includegraphics[width=.22\textwidth,height=.3\textheight,type=png,ext=.png,read=.png]{/afs/cern.ch/user/a/atpathak/afswork/public/Pixel/ALLTXT_22May2017_r9139/pic/mydata_InDetSiSpTrackFinderSLHC_1} 
\includegraphics[width=.22\textwidth,height=.3\textheight,type=png,ext=.png,read=.png]{/afs/cern.ch/user/a/atpathak/afswork/public/Pixel/ALLTXT_22May2017_r9139/pic/mydata_InDetSiSpTrackFinderSLHC_2}
\includegraphics[width=.22\textwidth,height=.3\textheight,type=png,ext=.png,read=.png]{/afs/cern.ch/user/a/atpathak/afswork/public/Pixel/ALLTXT_22May2017_r9139/pic/mydata_InDetSiSpTrackFinderSLHC_5} 
%\item Time (minute) / event 

for 30 sets of jobs, each over 10, 20, 50, 100  events (left to right)\\
\includegraphics[width=.22\textwidth,height=.3\textheight,type=png,ext=.png,read=.png]{/afs/cern.ch/user/a/atpathak/afswork/public/Pixel/ALLTXT_22May2017_r9139/pic/mydata_InDetSiSpTrackFinderSLHC_10}
\includegraphics[width=.22\textwidth,height=.3\textheight,type=png,ext=.png,read=.png]{/afs/cern.ch/user/a/atpathak/afswork/public/Pixel/ALLTXT_22May2017_r9139/pic/mydata_InDetSiSpTrackFinderSLHC_20}
\includegraphics[width=.22\textwidth,height=.3\textheight,type=png,ext=.png,read=.png]{/afs/cern.ch/user/a/atpathak/afswork/public/Pixel/ALLTXT_22May2017_r9139/pic/mydata_InDetSiSpTrackFinderSLHC_50} 
\includegraphics[width=.22\textwidth,height=.3\textheight,type=png,ext=.png,read=.png]{/afs/cern.ch/user/a/atpathak/afswork/public/Pixel/ALLTXT_22May2017_r9139/pic/mydata_InDetSiSpTrackFinderSLHC_100}  
\end{center}
%\item \\
%\end{itemize}
\end{normalsize}
\end{center}
\end{frame}
%------------------------------------------------
\begin{frame}
\frametitle{Times (min) for InDetSiSpTrackFinderSLHC for BCL (r9139)}
\begin{normalsize}
\vspace*{-0.6cm}
\begin{itemize}
\begin{small}
\item The following plots show (fitted mean and fitted error ) / per event as function of NumberOfEvents
\item Fits to a constant value for [1 to 100], [2 to 100], [5 to 100]
  and [10 to 100] events, respectively, are shown below from left to right: 
\end{small}
\end{itemize}
\vspace*{0.2cm}
\includegraphics[width=.27\textwidth,height=.3\textheight,type=png,ext=.png,read=.png]{/afs/cern.ch/user/a/atpathak/afswork/public/Pixel/ALLTXT_22May2017_r9139/InDetSiSpTrackFinderSLHC_LinearFit_1} 
\includegraphics[width=.27\textwidth,height=.3\textheight,type=png,ext=.png,read=.png]{/afs/cern.ch/user/a/atpathak/afswork/public/Pixel/ALLTXT_22May2017_r9139/InDetSiSpTrackFinderSLHC_LinearFit_2}
\includegraphics[width=.27\textwidth,height=.3\textheight,type=png,ext=.png,read=.png]{/afs/cern.ch/user/a/atpathak/afswork/public/Pixel/ALLTXT_22May2017_r9139/InDetSiSpTrackFinderSLHC_LinearFit_5}
\includegraphics[width=.27\textwidth,height=.3\textheight,type=png,ext=.png,read=.png]{/afs/cern.ch/user/a/atpathak/afswork/public/Pixel/ALLTXT_22May2017_r9139/InDetSiSpTrackFinderSLHC_LinearFit_10} 
\begin{itemize}
\begin{small}
\item Error (0.005104) from last fit $\sim$ Max - Min (= 0.002) of
  fitted means from the 4 sets of fits $\Rightarrow$ no additional 
  systematic error.
\end{small}
\end{itemize}
\end{normalsize}
\end{frame}
%------------------------------------------------
\begin{frame}
\begin{Large}
\centerline{Reconstruction times for InDetAmbiguitySolverSLHC}
\end{Large}
\end{frame}
%------------------------------------------------
\begin{frame}
\frametitle{Times (min) for InDetAmbiguitySolverSLHC for Long Strips}

\begin{center}
\begin{normalsize}

%\begin{itemize}

%\item Time (minute) / event 
for 30 sets of jobs, each over 1, 2, 5 events (left to right) \\
%\item Timing for r9060 [left], r9133 [center] and r9139 [right]\\
%\item \\
%\item \\
\vspace*{-.2cm}
\begin{center}
%\item For r9060(left) and For r9133(right) \\
\includegraphics[width=.22\textwidth,height=.3\textheight,type=png,ext=.png,read=.png]{/afs/cern.ch/user/a/atpathak/afswork/public/Pixel/ALLTXT_22May2017_r9133/pic/mydata_InDetAmbiguitySolverSLHC_1} 
\includegraphics[width=.22\textwidth,height=.3\textheight,type=png,ext=.png,read=.png]{/afs/cern.ch/user/a/atpathak/afswork/public/Pixel/ALLTXT_22May2017_r9133/pic/mydata_InDetAmbiguitySolverSLHC_2}
\includegraphics[width=.22\textwidth,height=.3\textheight,type=png,ext=.png,read=.png]{/afs/cern.ch/user/a/atpathak/afswork/public/Pixel/ALLTXT_22May2017_r9133/pic/mydata_InDetAmbiguitySolverSLHC_5} 
%\item Time (minute) / event 

for 30 sets of jobs, each over 10, 20, 50, 100  events (left to right)\\
\includegraphics[width=.22\textwidth,height=.3\textheight,type=png,ext=.png,read=.png]{/afs/cern.ch/user/a/atpathak/afswork/public/Pixel/ALLTXT_22May2017_r9133/pic/mydata_InDetAmbiguitySolverSLHC_10}
\includegraphics[width=.22\textwidth,height=.3\textheight,type=png,ext=.png,read=.png]{/afs/cern.ch/user/a/atpathak/afswork/public/Pixel/ALLTXT_22May2017_r9133/pic/mydata_InDetAmbiguitySolverSLHC_20}
\includegraphics[width=.22\textwidth,height=.3\textheight,type=png,ext=.png,read=.png]{/afs/cern.ch/user/a/atpathak/afswork/public/Pixel/ALLTXT_22May2017_r9133/pic/mydata_InDetAmbiguitySolverSLHC_50} 
\includegraphics[width=.22\textwidth,height=.3\textheight,type=png,ext=.png,read=.png]{/afs/cern.ch/user/a/atpathak/afswork/public/Pixel/ALLTXT_22May2017_r9133/pic/mydata_InDetAmbiguitySolverSLHC_100}  
\end{center}
%\item \\
%\end{itemize}
\end{normalsize}
\end{center}
\end{frame}
%------------------------------------------------
\begin{frame}
\frametitle{Times (min) for InDetAmbiguitySolverSLHC for Long Strips}
\begin{normalsize}
\vspace*{-0.6cm}
\begin{itemize}
\begin{small}
\item The following plots show (fitted mean and fitted error ) / per event as function of NumberOfEvents
\item Fits to a constant value for [1 to 100], [2 to 100], [5 to 100]
  and [10 to 100] events, respectively, are shown below from left to right: 
\end{small}
\end{itemize}
\vspace*{0.2cm}
\includegraphics[width=.27\textwidth,height=.3\textheight,type=png,ext=.png,read=.png]{/afs/cern.ch/user/a/atpathak/afswork/public/Pixel/ALLTXT_22May2017_r9133/InDetAmbiguitySolverSLHC_LinearFit_1} 
\includegraphics[width=.27\textwidth,height=.3\textheight,type=png,ext=.png,read=.png]{/afs/cern.ch/user/a/atpathak/afswork/public/Pixel/ALLTXT_22May2017_r9133/InDetAmbiguitySolverSLHC_LinearFit_2}
\includegraphics[width=.27\textwidth,height=.3\textheight,type=png,ext=.png,read=.png]{/afs/cern.ch/user/a/atpathak/afswork/public/Pixel/ALLTXT_22May2017_r9133/InDetAmbiguitySolverSLHC_LinearFit_5}
\includegraphics[width=.27\textwidth,height=.3\textheight,type=png,ext=.png,read=.png]{/afs/cern.ch/user/a/atpathak/afswork/public/Pixel/ALLTXT_22May2017_r9133/InDetAmbiguitySolverSLHC_LinearFit_10} 
\begin{itemize}
\begin{small}
\item Error (0.001137) from last fit $\sim$ Max - Min (=0.0004) of
  fitted means from the 4 sets of fits $\Rightarrow$ no additional 
  systematic error.
\end{small}
\end{itemize}
\end{normalsize}
\end{frame}
%------------------------------------------------
\begin{frame}
\frametitle{Times (min) for InDetAmbiguitySolverSLHC for BCL (r9139)}

\begin{center}
\begin{normalsize}

%\begin{itemize}

%\item Time (minute) / event 
for 30 sets of jobs, each over 1, 2, 5 events (left to right) \\
%\item Timing for r9060 [left], r9133 [center] and r9139 [right]\\
%\item \\
%\item \\
\vspace*{-.2cm}
\begin{center}
%\item For r9060(left) and For r9133(right) \\
\includegraphics[width=.22\textwidth,height=.3\textheight,type=png,ext=.png,read=.png]{/afs/cern.ch/user/a/atpathak/afswork/public/Pixel/ALLTXT_22May2017_r9139/pic/mydata_InDetAmbiguitySolverSLHC_1} 
\includegraphics[width=.22\textwidth,height=.3\textheight,type=png,ext=.png,read=.png]{/afs/cern.ch/user/a/atpathak/afswork/public/Pixel/ALLTXT_22May2017_r9139/pic/mydata_InDetAmbiguitySolverSLHC_2}
\includegraphics[width=.22\textwidth,height=.3\textheight,type=png,ext=.png,read=.png]{/afs/cern.ch/user/a/atpathak/afswork/public/Pixel/ALLTXT_22May2017_r9139/pic/mydata_InDetAmbiguitySolverSLHC_5} 
%\item Time (minute) / event 

for 30 sets of jobs, each over 10, 20, 50, 100  events (left to right)\\
\includegraphics[width=.22\textwidth,height=.3\textheight,type=png,ext=.png,read=.png]{/afs/cern.ch/user/a/atpathak/afswork/public/Pixel/ALLTXT_22May2017_r9139/pic/mydata_InDetAmbiguitySolverSLHC_10}
\includegraphics[width=.22\textwidth,height=.3\textheight,type=png,ext=.png,read=.png]{/afs/cern.ch/user/a/atpathak/afswork/public/Pixel/ALLTXT_22May2017_r9139/pic/mydata_InDetAmbiguitySolverSLHC_20}
\includegraphics[width=.22\textwidth,height=.3\textheight,type=png,ext=.png,read=.png]{/afs/cern.ch/user/a/atpathak/afswork/public/Pixel/ALLTXT_22May2017_r9139/pic/mydata_InDetAmbiguitySolverSLHC_50} 
\includegraphics[width=.22\textwidth,height=.3\textheight,type=png,ext=.png,read=.png]{/afs/cern.ch/user/a/atpathak/afswork/public/Pixel/ALLTXT_22May2017_r9139/pic/mydata_InDetAmbiguitySolverSLHC_100}  
\end{center}
%\item \\
%\end{itemize}
\end{normalsize}
\end{center}
\end{frame}
%------------------------------------------------
\begin{frame}
\frametitle{For InDetAmbiguitySolverSLHC for BCL (r9139)}
\begin{normalsize}
\vspace*{-0.6cm}
\begin{itemize}
\begin{small}
\item The following plots show (fitted mean and fitted error ) / per event as function of NumberOfEvents
\item Fits to a constant value for [1 to 100], [2 to 100], [5 to 100]
  and [10 to 100] events, respectively, are shown below from left to right: 
\end{small}
\end{itemize}
\vspace*{0.2cm}
\includegraphics[width=.27\textwidth,height=.3\textheight,type=png,ext=.png,read=.png]{/afs/cern.ch/user/a/atpathak/afswork/public/Pixel/ALLTXT_22May2017_r9139/InDetAmbiguitySolverSLHC_LinearFit_1} 
\includegraphics[width=.27\textwidth,height=.3\textheight,type=png,ext=.png,read=.png]{/afs/cern.ch/user/a/atpathak/afswork/public/Pixel/ALLTXT_22May2017_r9139/InDetAmbiguitySolverSLHC_LinearFit_2}
\includegraphics[width=.27\textwidth,height=.3\textheight,type=png,ext=.png,read=.png]{/afs/cern.ch/user/a/atpathak/afswork/public/Pixel/ALLTXT_22May2017_r9139/InDetAmbiguitySolverSLHC_LinearFit_5}
\includegraphics[width=.27\textwidth,height=.3\textheight,type=png,ext=.png,read=.png]{/afs/cern.ch/user/a/atpathak/afswork/public/Pixel/ALLTXT_22May2017_r9139/InDetAmbiguitySolverSLHC_LinearFit_10} 
\begin{itemize}
\begin{small}
\item Error (0.001144) from last fit $\sim$ Max - Min (= 0.0002) of
  fitted means from the 4 sets of fits $\Rightarrow$ no additional 
  systematic error.
\end{small}
\end{itemize}
\end{normalsize}
\end{frame}
%------------------------------------------------
\begin{frame}
\begin{Large}
\centerline{Reconstruction times for InDetSCT\_Clusterization}
\end{Large}
\end{frame}
%------------------------------------------------
\begin{frame}
\frametitle{Times (Sec) for InDetSCT\_Clusterization for Long Strips}

\begin{center}
\begin{normalsize}

%\begin{itemize}

%\item Time (Second) / event  
for 30 sets of jobs, each over 1, 2, 5 events (left to right)\\
%\item Timing for r9060 [left], r9133 [center] and r9139 [right]\\
%\item \\
%\item \\
\vspace*{-.2cm}
\begin{center}
%\item For r9060(left) and For r9133(right) \\
\includegraphics[width=.22\textwidth,height=.3\textheight,type=png,ext=.png,read=.png]{/afs/cern.ch/user/a/atpathak/afswork/public/Pixel/ALLTXT_22May2017_r9133/pic/mydata_InDetSCTClusterization_1} 
\includegraphics[width=.22\textwidth,height=.3\textheight,type=png,ext=.png,read=.png]{/afs/cern.ch/user/a/atpathak/afswork/public/Pixel/ALLTXT_22May2017_r9133/pic/mydata_InDetSCTClusterization_2}
\includegraphics[width=.22\textwidth,height=.3\textheight,type=png,ext=.png,read=.png]{/afs/cern.ch/user/a/atpathak/afswork/public/Pixel/ALLTXT_22May2017_r9133/pic/mydata_InDetSCTClusterization_5} 

%\item Time (Second) / event 

for 30 sets of jobs, each over 10, 20, 50, 100  events (left to right) \\
\includegraphics[width=.22\textwidth,height=.3\textheight,type=png,ext=.png,read=.png]{/afs/cern.ch/user/a/atpathak/afswork/public/Pixel/ALLTXT_22May2017_r9133/pic/mydata_InDetSCTClusterization_10}
\includegraphics[width=.22\textwidth,height=.3\textheight,type=png,ext=.png,read=.png]{/afs/cern.ch/user/a/atpathak/afswork/public/Pixel/ALLTXT_22May2017_r9133/pic/mydata_InDetSCTClusterization_20}
\includegraphics[width=.22\textwidth,height=.3\textheight,type=png,ext=.png,read=.png]{/afs/cern.ch/user/a/atpathak/afswork/public/Pixel/ALLTXT_22May2017_r9133/pic/mydata_InDetSCTClusterization_50} 
\includegraphics[width=.22\textwidth,height=.3\textheight,type=png,ext=.png,read=.png]{/afs/cern.ch/user/a/atpathak/afswork/public/Pixel/ALLTXT_22May2017_r9133/pic/mydata_InDetSCTClusterization_100}  
\end{center}
%\item \\
%\end{itemize}
\end{normalsize}
\end{center}
\end{frame}
%------------------------------------------------
\begin{frame}
\frametitle{Times (Sec) for InDetSCT\_Clusterization for Long Strips}
\begin{normalsize}
\vspace*{-0.6cm}
\begin{itemize}
\begin{small}
\item The following plots show (fitted mean and fitted error ) / per event as function of NumberOfEvents
\item Fits to a constant value for [1 to 100], [2 to 100], [5 to 100]
  and [10 to 100] events, respectively, are shown below from left to right: 
\end{small}
\end{itemize}
\vspace*{0.2cm}
\includegraphics[width=.27\textwidth,height=.3\textheight,type=png,ext=.png,read=.png]{/afs/cern.ch/user/a/atpathak/afswork/public/Pixel/ALLTXT_22May2017_r9133/InDetSCTClusterization_LinearFit_1} 
\includegraphics[width=.27\textwidth,height=.3\textheight,type=png,ext=.png,read=.png]{/afs/cern.ch/user/a/atpathak/afswork/public/Pixel/ALLTXT_22May2017_r9133/InDetSCTClusterization_LinearFit_2}
\includegraphics[width=.27\textwidth,height=.3\textheight,type=png,ext=.png,read=.png]{/afs/cern.ch/user/a/atpathak/afswork/public/Pixel/ALLTXT_22May2017_r9133/InDetSCTClusterization_LinearFit_5}
\includegraphics[width=.27\textwidth,height=.3\textheight,type=png,ext=.png,read=.png]{/afs/cern.ch/user/a/atpathak/afswork/public/Pixel/ALLTXT_22May2017_r9133/InDetSCTClusterization_LinearFit_10} 
\begin{itemize}
\begin{small}
\item Error (0.001216) from last fit $\sim$  Max - Min (=0.0012 ) of
  fitted means from the 4 sets of fits $\Rightarrow$ no additional 
  systematic error.
\end{small}
\end{itemize}
\end{normalsize}
\end{frame}
%------------------------------------------------
\begin{frame}
\frametitle{Times (Sec) for InDetSCT\_Clusterization for BCL (r9139)}

\begin{center}
\begin{normalsize}

%\begin{itemize}

%\item Time (Second) / event 
for 30 sets of jobs, each over 1, 2, 5 events (left to right)\\
%\item Timing for r9060 [left], r9133 [center] and r9139 [right]\\
%\item \\
%\item \\
\vspace*{-.2cm}
\begin{center}
%\item For r9060(left) and For r9133(right) \\
\includegraphics[width=.22\textwidth,height=.3\textheight,type=png,ext=.png,read=.png]{/afs/cern.ch/user/a/atpathak/afswork/public/Pixel/ALLTXT_22May2017_r9139/pic/mydata_InDetSCTClusterization_1} 
\includegraphics[width=.22\textwidth,height=.3\textheight,type=png,ext=.png,read=.png]{/afs/cern.ch/user/a/atpathak/afswork/public/Pixel/ALLTXT_22May2017_r9139/pic/mydata_InDetSCTClusterization_2}
\includegraphics[width=.22\textwidth,height=.3\textheight,type=png,ext=.png,read=.png]{/afs/cern.ch/user/a/atpathak/afswork/public/Pixel/ALLTXT_22May2017_r9139/pic/mydata_InDetSCTClusterization_5} 

%\item Time (Second) / event 

for 30 sets of jobs, each over 10, 20, 50, 100  events (left to right) \\
\includegraphics[width=.22\textwidth,height=.3\textheight,type=png,ext=.png,read=.png]{/afs/cern.ch/user/a/atpathak/afswork/public/Pixel/ALLTXT_22May2017_r9139/pic/mydata_InDetSCTClusterization_10}
\includegraphics[width=.22\textwidth,height=.3\textheight,type=png,ext=.png,read=.png]{/afs/cern.ch/user/a/atpathak/afswork/public/Pixel/ALLTXT_22May2017_r9139/pic/mydata_InDetSCTClusterization_20}
\includegraphics[width=.22\textwidth,height=.3\textheight,type=png,ext=.png,read=.png]{/afs/cern.ch/user/a/atpathak/afswork/public/Pixel/ALLTXT_22May2017_r9139/pic/mydata_InDetSCTClusterization_50} 
\includegraphics[width=.22\textwidth,height=.3\textheight,type=png,ext=.png,read=.png]{/afs/cern.ch/user/a/atpathak/afswork/public/Pixel/ALLTXT_22May2017_r9139/pic/mydata_InDetSCTClusterization_100}  
\end{center}
%\item \\
%\end{itemize}
\end{normalsize}
\end{center}
\end{frame}

%------------------------------------------------
\begin{frame}
\frametitle{Times (Sec) for InDetSCT\_Clusterization for BCL(r9139)}
\begin{normalsize}
\vspace*{-0.6cm}
\begin{itemize}
\begin{small}
\item The following plots show (fitted mean and fitted error ) / per event as function of NumberOfEvents
\item Fits to a constant value for [1 to 100], [2 to 100], [5 to 100]
  and [10 to 100] events, respectively, are shown below from left to right: 
\end{small}
\end{itemize}
\vspace*{0.2cm}
\includegraphics[width=.27\textwidth,height=.3\textheight,type=png,ext=.png,read=.png]{/afs/cern.ch/user/a/atpathak/afswork/public/Pixel/ALLTXT_22May2017_r9139/InDetSCTClusterization_LinearFit_1} 
\includegraphics[width=.27\textwidth,height=.3\textheight,type=png,ext=.png,read=.png]{/afs/cern.ch/user/a/atpathak/afswork/public/Pixel/ALLTXT_22May2017_r9139/InDetSCTClusterization_LinearFit_2}
\includegraphics[width=.27\textwidth,height=.3\textheight,type=png,ext=.png,read=.png]{/afs/cern.ch/user/a/atpathak/afswork/public/Pixel/ALLTXT_22May2017_r9139/InDetSCTClusterization_LinearFit_5}
\includegraphics[width=.27\textwidth,height=.3\textheight,type=png,ext=.png,read=.png]{/afs/cern.ch/user/a/atpathak/afswork/public/Pixel/ALLTXT_22May2017_r9139/InDetSCTClusterization_LinearFit_10} 
\begin{itemize}
\begin{small}
\item Error (0.001418) from last fit $\sim$  Max - Min (=0.0022 ) of
  fitted means from the 4 sets of fits $\Rightarrow$ no additional 
  systematic error.
\end{small}
\end{itemize}
\end{normalsize}
\end{frame}
%------------------------------------------------
\end{document}
%------------------------------------------------
\begin{frame}
\frametitle{Fitted mean per event vs Number of events for InDetSiSpTrackFinderSLHC for Nominal (r9)}

%\begin{center}
\begin{normalsize}

%\begin{itemize}

%\item Time (minute) / event 
%For AthAlgSeq (left)  and InDetSiSpTrackFinderSLHC (right) \\
%\item Timing for r9060 [left], r9133 [center] and r9139 [right]\\
%\item \\
%\item \\
\vspace*{-0.6cm}
\begin{center}
%\item For r9060(left) and For r9133(right) \\
%\includegraphics[width=.44\textwidth,height=.32\textheight,type=png,ext=.png,read=.png]{/afs/cern.ch/user/a/atpathak/afswork/public/Pixel/ALLTXT_22May2017_r9060/AthAlgSeq_LinearFit} 
\includegraphics[width=.55\textwidth,height=.42\textheight,type=png,ext=.png,read=.png]{/afs/cern.ch/user/a/atpathak/afswork/public/Pixel/ALLTXT_22May2017_r9060/InDetSiSpTrackFinderSLHC_LinearFit}
\end{center}
%\item Time (minute) / event
%\begin{small} 
%For InDetAmbiguitySolverSLHC(left) and InDetSCT\_Clusterization(right)\\
%\end{small}
%\begin{center}
%\includegraphics[width=.44\textwidth,height=.32\textheight,type=png,ext=.png,read=.png]{/afs/cern.ch/user/a/atpathak/afswork/public/Pixel/ALLTXT_22May2017_r9060/InDetAmbiguitySolverSLHC_LinearFit}
%\includegraphics[width=.44\textwidth,height=.32\textheight,type=png,ext=.png,read=.png]{/afs/cern.ch/user/a/atpathak/afswork/public/Pixel/ALLTXT_22May2017_r9060/InDetSCTClusterization_LinearFit} 

%\end{center}
%\item \\
%\end{itemize}
\vspace*{-0.6cm}
\begin{itemize}
\begin{small}
\item This plot shows that the linear distribution between Number of events
  and Fitted mean/event for InDetSiSpTrackFinderSLHC 
\item The probability for the p0 fit is less than the p1 fit that
  indicates p0 is the best fit as compared to p1 
\item Since p0 is the best fit, we can conclude that Total Time taken
  per event is independent of the number of events.  
%\item The following 3 different configurations are studied:
\end{small}
\end{itemize}
\end{normalsize}
%\end{center}
\end{frame}
%------------------------------------------------
\begin{frame}
\frametitle{Fitted mean per event vs Number of events for InDetAmbiguitySolverSLHC for Nominal (r9060)}

%\begin{center}
\begin{normalsize}

%\begin{itemize}

%\item Time (minute) / event 
%For AthAlgSeq (left)  and InDetSiSpTrackFinderSLHC (right) \\
%\item Timing for r9060 [left], r9133 [center] and r9139 [right]\\
%\item \\
%\item \\
\vspace*{-0.6cm}
%\begin{center}
%\item For r9060(left) and For r9133(right) \\
%\includegraphics[width=.44\textwidth,height=.32\textheight,type=png,ext=.png,read=.png]{/afs/cern.ch/user/a/atpathak/afswork/public/Pixel/ALLTXT_22May2017_r9060/AthAlgSeq_LinearFit} 
%\includegraphics[width=.55\textwidth,height=.42\textheight,type=png,ext=.png,read=.png]{/afs/cern.ch/user/a/atpathak/afswork/public/Pixel/ALLTXT_22May2017_r9060/InDetSiSpTrackFinderSLHC_LinearFit}
%\end{center}
%\item Time (minute) / event
%\begin{small} 
%For InDetAmbiguitySolverSLHC(left) and InDetSCT\_Clusterization(right)\\
%\end{small}
\begin{center}
\includegraphics[width=.55\textwidth,height=.42\textheight,type=png,ext=.png,read=.png]{/afs/cern.ch/user/a/atpathak/afswork/public/Pixel/ALLTXT_22May2017_r9060/InDetAmbiguitySolverSLHC_LinearFit}
%\includegraphics[width=.44\textwidth,height=.32\textheight,type=png,ext=.png,read=.png]{/afs/cern.ch/user/a/atpathak/afswork/public/Pixel/ALLTXT_22May2017_r9060/InDetSCTClusterization_LinearFit} 

\end{center}
%\item \\
%\end{itemize}
\vspace*{-0.6cm}
\begin{itemize}
\begin{small}
\item This plot shows that the linear distribution between Number of events
  and Fitted mean/event for InDetAmbiguitySolverSLHC 
\item The probability for the p0 fit is less than the p1 fit that
  indicates p0 is the best fit as compared to p1 
\item Since p0 is the best fit, we can conclude that Total Time taken
  per event is independent of the number of events.  
%\item The following 3 different configurations are studied:
\end{small}
\end{itemize}
\end{normalsize}
%\end{center}
\end{frame}
%-----------------------------------------------------
\begin{frame}
\frametitle{Fitted mean per event vs Number of events for InDetSCT\_Clusterization for Nominal (r9060)}

%\begin{center}
\begin{normalsize}

%\begin{itemize}

%\item Time (minute) / event 
%For AthAlgSeq (left)  and InDetSiSpTrackFinderSLHC (right) \\
%\item Timing for r9060 [left], r9133 [center] and r9139 [right]\\
%\item \\
%\item \\
\vspace*{-0.6cm}
%\begin{center}
%\item For r9060(left) and For r9133(right) \\
%\includegraphics[width=.44\textwidth,height=.32\textheight,type=png,ext=.png,read=.png]{/afs/cern.ch/user/a/atpathak/afswork/public/Pixel/ALLTXT_22May2017_r9060/AthAlgSeq_LinearFit} 
%\includegraphics[width=.55\textwidth,height=.42\textheight,type=png,ext=.png,read=.png]{/afs/cern.ch/user/a/atpathak/afswork/public/Pixel/ALLTXT_22May2017_r9060/InDetSiSpTrackFinderSLHC_LinearFit}
%\end{center}
%\item Time (minute) / event
%\begin{small} 
%For InDetAmbiguitySolverSLHC(left) and InDetSCT\_Clusterization(right)\\
%\end{small}
\begin{center}
%\includegraphics[width=.55\textwidth,height=.42\textheight,type=png,ext=.png,read=.png]{/afs/cern.ch/user/a/atpathak/afswork/public/Pixel/ALLTXT_22May2017_r9060/InDetAmbiguitySolverSLHC_LinearFit}
\includegraphics[width=.55\textwidth,height=.42\textheight,type=png,ext=.png,read=.png]{/afs/cern.ch/user/a/atpathak/afswork/public/Pixel/ALLTXT_22May2017_r9060/InDetSCTClusterization_LinearFit} 

\end{center}
%\item \\
%\end{itemize}
\vspace*{-0.6cm}
\begin{itemize}
\begin{small}
\item This plot shows that the linear distribution between Number of events
  and Fitted mean/event for InDetSCT\_Clusterization 
\item The probability for the p0 fit is less than the p1 fit that
  indicates p0 is the best fit as compared to p1 
\item Since p0 is the best fit, we can conclude that Total Time taken
  per event is independent of the number of events.  
%\item The following 3 different configurations are studied:
\end{small}
\end{itemize}
\end{normalsize}
%\end{center}
\end{frame}
%------------------------------------------------
\begin{frame}
\frametitle{Fitted mean per event vs Number of events in Long Strips(r9133) for AthAlgSeq}

%\begin{center}
\begin{normalsize}

%\begin{itemize}

%\item Time (minute) / event 
%For AthAlgSeq (left)  and InDetSiSpTrackFinderSLHC (right) \\
%\item Timing for r9060 [left], r9133 [center] and r9139 [right]\\
%\item \\
%\item \\
\vspace*{-0.6cm}
\begin{center}
%\item For r9060(left) and For r9133(right) \\
\includegraphics[width=.55\textwidth,height=.42\textheight,type=png,ext=.png,read=.png]{/afs/cern.ch/user/a/atpathak/afswork/public/Pixel/ALLTXT_22May2017_r9133/AthAlgSeq_LinearFit} 
%\includegraphics[width=.44\textwidth,height=.32\textheight,type=png,ext=.png,read=.png]{/afs/cern.ch/user/a/atpathak/afswork/public/Pixel/ALLTXT_22May2017_r9133/InDetSiSpTrackFinderSLHC_LinearFit}
\end{center}
%\item Time (minute) / event 
%\begin{small} 
%for InDetAmbiguitySolverSLHC(left) and InDetSCT\_Clusterization(right)\\
%\end{small}
%\begin{center}
%\includegraphics[width=.44\textwidth,height=.32\textheight,type=png,ext=.png,read=.png]{/afs/cern.ch/user/a/atpathak/afswork/public/Pixel/ALLTXT_22May2017_r9133/InDetAmbiguitySolverSLHC_LinearFit}
%\includegraphics[width=.44\textwidth,height=.32\textheight,type=png,ext=.png,read=.png]{/afs/cern.ch/user/a/atpathak/afswork/public/Pixel/ALLTXT_22May2017_r9133/InDetSCTClusterization_LinearFit} 

%\end{center}
%\item \\
%\end{itemize}
\vspace*{-0.6cm}
\begin{itemize}
\begin{small}
\item This plot shows that the linear distribution between Number of events
  and Fitted mean/event for AthAlgSeq 
\item The probability for the p0 fit is less than the p1 fit that
  indicates p0 is the best fit as compared to p1 
\item Since p0 is the best fit, we can conclude that Total Time taken
  per event is independent of number of events.  
%\item The following 3 different configurations are studied:
\end{small}
\end{itemize}
\end{normalsize}
%\end{center}
\end{frame}
%------------------------------------------------
\begin{frame}
\frametitle{Fitted mean per event vs Number of events in Long Strips(r9133) for InDetSiSpTrackFinderSLHC}

%\begin{center}
\begin{normalsize}

%\begin{itemize}

%\item Time (minute) / event 
%For AthAlgSeq (left)  and InDetSiSpTrackFinderSLHC (right) \\
%\item Timing for r9060 [left], r9133 [center] and r9139 [right]\\
%\item \\
%\item \\
\vspace*{-0.6cm}
\begin{center}
%\item For r9060(left) and For r9133(right) \\
%\includegraphics[width=.55\textwidth,height=.42\textheight,type=png,ext=.png,read=.png]{/afs/cern.ch/user/a/atpathak/afswork/public/Pixel/ALLTXT_22May2017_r9133/AthAlgSeq_LinearFit} 
\includegraphics[width=.55\textwidth,height=.42\textheight,type=png,ext=.png,read=.png]{/afs/cern.ch/user/a/atpathak/afswork/public/Pixel/ALLTXT_22May2017_r9133/InDetSiSpTrackFinderSLHC_LinearFit}
\end{center}
%\item Time (minute) / event 
%\begin{small} 
%for InDetAmbiguitySolverSLHC(left) and InDetSCT\_Clusterization(right)\\
%\end{small}
%\begin{center}
%\includegraphics[width=.44\textwidth,height=.32\textheight,type=png,ext=.png,read=.png]{/afs/cern.ch/user/a/atpathak/afswork/public/Pixel/ALLTXT_22May2017_r9133/InDetAmbiguitySolverSLHC_LinearFit}
%\includegraphics[width=.44\textwidth,height=.32\textheight,type=png,ext=.png,read=.png]{/afs/cern.ch/user/a/atpathak/afswork/public/Pixel/ALLTXT_22May2017_r9133/InDetSCTClusterization_LinearFit} 

%\end{center}
%\item \\
%\end{itemize}
\vspace*{-0.6cm}
\begin{itemize}
\begin{small}
\item This plot shows that the linear distribution between Number of events
  and Fitted mean/event for InDetSiSpTrackFinderSLHC 
\item The probability for the p0 fit is less than the p1 fit that
  indicates p0 is the best fit as compared to p1 
\item Since p0 is the best fit, we can conclude that Total Time taken
  per event is independent of the number of events.  
%\item The following 3 different configurations are studied:
\end{small}
\end{itemize}
\end{normalsize}
%\end{center}
\end{frame}
%------------------------------------------------
\begin{frame}
\frametitle{Fitted mean per event vs Number of events in Long Strips(r9133) for InDetAmbiguitySolverSLHC}

%\begin{center}
\begin{normalsize}

%\begin{itemize}

%\item Time (minute) / event 
%For AthAlgSeq (left)  and InDetSiSpTrackFinderSLHC (right) \\
%\item Timing for r9060 [left], r9133 [center] and r9139 [right]\\
%\item \\
%\item \\
\vspace*{-0.6cm}
%\begin{center}
%\item For r9060(left) and For r9133(right) \\
%\includegraphics[width=.55\textwidth,height=.42\textheight,type=png,ext=.png,read=.png]{/afs/cern.ch/user/a/atpathak/afswork/public/Pixel/ALLTXT_22May2017_r9133/AthAlgSeq_LinearFit} 
%\includegraphics[width=.44\textwidth,height=.32\textheight,type=png,ext=.png,read=.png]{/afs/cern.ch/user/a/atpathak/afswork/public/Pixel/ALLTXT_22May2017_r9133/InDetSiSpTrackFinderSLHC_LinearFit}
%\end{center}
%\item Time (minute) / event 
%\begin{small} 
%for InDetAmbiguitySolverSLHC(left) and InDetSCT\_Clusterization(right)\\
%\end{small}
\begin{center}
\includegraphics[width=.55\textwidth,height=.42\textheight,type=png,ext=.png,read=.png]{/afs/cern.ch/user/a/atpathak/afswork/public/Pixel/ALLTXT_22May2017_r9133/InDetAmbiguitySolverSLHC_LinearFit}
%\includegraphics[width=.44\textwidth,height=.32\textheight,type=png,ext=.png,read=.png]{/afs/cern.ch/user/a/atpathak/afswork/public/Pixel/ALLTXT_22May2017_r9133/InDetSCTClusterization_LinearFit} 

\end{center}
%\item \\
%\end{itemize}
\vspace*{-0.6cm}
\begin{itemize}
\begin{small}
\item This plot shows that the linear distribution between Number of events
  and Fitted mean/event for InDetAmbiguitySolverSLHC 
\item The probability for the p0 fit is less than the p1 fit that
  indicates p0 is the best fit as compared to p1 
\item Since p0 is the best fit, we can conclude that Total Time taken
  per event is independent of the number of events.  
%\item The following 3 different configurations are studied:
\end{small}
\end{itemize}
\end{normalsize}
%\end{center}
\end{frame}
%------------------------------------------------
\begin{frame}
\frametitle{Fitted mean per event vs Number of events in Long Strips(r9133) for InDetSCT\_Clusterization}

%\begin{center}
\begin{normalsize}

%\begin{itemize}

%\item Time (minute) / event 
%For AthAlgSeq (left)  and InDetSiSpTrackFinderSLHC (right) \\
%\item Timing for r9060 [left], r9133 [center] and r9139 [right]\\
%\item \\
%\item \\
\vspace*{-0.6cm}
%\begin{center}
%\item For r9060(left) and For r9133(right) \\
%\includegraphics[width=.55\textwidth,height=.42\textheight,type=png,ext=.png,read=.png]{/afs/cern.ch/user/a/atpathak/afswork/public/Pixel/ALLTXT_22May2017_r9133/AthAlgSeq_LinearFit} 
%\includegraphics[width=.44\textwidth,height=.32\textheight,type=png,ext=.png,read=.png]{/afs/cern.ch/user/a/atpathak/afswork/public/Pixel/ALLTXT_22May2017_r9133/InDetSiSpTrackFinderSLHC_LinearFit}
%\end{center}
%\item Time (minute) / event 
%\begin{small} 
%for InDetAmbiguitySolverSLHC(left) and InDetSCT\_Clusterization(right)\\
%\end{small}
\begin{center}
%\includegraphics[width=.44\textwidth,height=.32\textheight,type=png,ext=.png,read=.png]{/afs/cern.ch/user/a/atpathak/afswork/public/Pixel/ALLTXT_22May2017_r9133/InDetAmbiguitySolverSLHC_LinearFit}
\includegraphics[width=.55\textwidth,height=.42\textheight,type=png,ext=.png,read=.png]{/afs/cern.ch/user/a/atpathak/afswork/public/Pixel/ALLTXT_22May2017_r9133/InDetSCTClusterization_LinearFit} 

\end{center}
%\item \\
%\end{itemize}
\vspace*{-0.6cm}
\begin{itemize}
\begin{small}
\item This plot shows that the linear distribution between Number of events
  and Fitted mean/event for InDetSCT\_Clusterization 
\item The probability for the p0 fit is less than the p1 fit that
  indicates p0 is the best fit as compared to p1 
\item Since p0 is the best fit, we can conclude that Total Time taken
  per event is independent of the number of events.  
%\item The following 3 different configurations are studied:
\end{small}
\end{itemize}
\end{normalsize}
%\end{center}
\end{frame}
%------------------------------------------------
\begin{frame}
\frametitle{Fitted mean per event vs Number of events in BCL(r9139) for AthAlgSeq}

%\begin{center}
\begin{normalsize}

%\begin{itemize}

%\item Time (minute) / event 
%For AthAlgSeq (left)  and InDetSiSpTrackFinderSLHC (right) \\
%\item Timing for r9060 [left], r9133 [center] and r9139 [right]\\
%\item \\
%\item \\
\vspace*{-0.6cm}
\begin{center}
%\item For r9060(left) and For r9133(right) \\
\includegraphics[width=.55\textwidth,height=.42\textheight,type=png,ext=.png,read=.png]{/afs/cern.ch/user/a/atpathak/afswork/public/Pixel/ALLTXT_22May2017_r9139/AthAlgSeq_LinearFit} 
%\includegraphics[width=.44\textwidth,height=.32\textheight,type=png,ext=.png,read=.png]{/afs/cern.ch/user/a/atpathak/afswork/public/Pixel/ALLTXT_22May2017_r9139/InDetSiSpTrackFinderSLHC_LinearFit}
\end{center}
%\item Time (minute) / event 
%\begin{small} 
%for InDetAmbiguitySolverSLHC(left) and InDetSCT\_Clusterization(right)\\
%\end{small}
%\begin{center}
%\includegraphics[width=.44\textwidth,height=.32\textheight,type=png,ext=.png,read=.png]{/afs/cern.ch/user/a/atpathak/afswork/public/Pixel/ALLTXT_22May2017_r9139/InDetAmbiguitySolverSLHC_LinearFit}
%\includegraphics[width=.44\textwidth,height=.32\textheight,type=png,ext=.png,read=.png]{/afs/cern.ch/user/a/atpathak/afswork/public/Pixel/ALLTXT_22May2017_r9139/InDetSCTClusterization_LinearFit} 

%\end{center}
%\item \\
%\end{itemize}
\vspace*{-0.6cm}
\begin{itemize}
\begin{small}
\item This plot shows that the linear distribution between Number of events
  and Fitted mean/event for AthAlgSeq 
\item The probability for the p0 fit is less than the p1 fit that
  indicates p0 is the best fit as compared to p1 
\item Since p0 is the best fit, we can conclude that Total Time taken
  per event is independent of the number of events.  
%\item The following 3 different configurations are studied:
\end{small}
\end{itemize}
\end{normalsize}
%\end{center}
\end{frame}
%------------------------------------------------
\begin{frame}
\frametitle{Fitted mean per event vs Number of events in BCL(r9139) for InDetSiSpTrackFinderSLHC}

%\begin{center}
\begin{normalsize}

%\begin{itemize}

%\item Time (minute) / event 
%For AthAlgSeq (left)  and InDetSiSpTrackFinderSLHC (right) \\
%\item Timing for r9060 [left], r9133 [center] and r9139 [right]\\
%\item \\
%\item \\
\vspace*{-0.6cm}
\begin{center}
%\item For r9060(left) and For r9133(right) \\
%\includegraphics[width=.55\textwidth,height=.42\textheight,type=png,ext=.png,read=.png]{/afs/cern.ch/user/a/atpathak/afswork/public/Pixel/ALLTXT_22May2017_r9139/AthAlgSeq_LinearFit} 
\includegraphics[width=.55\textwidth,height=.42\textheight,type=png,ext=.png,read=.png]{/afs/cern.ch/user/a/atpathak/afswork/public/Pixel/ALLTXT_22May2017_r9139/InDetSiSpTrackFinderSLHC_LinearFit}
\end{center}
%\item Time (minute) / event 
%\begin{small} 
%for InDetAmbiguitySolverSLHC(left) and InDetSCT\_Clusterization(right)\\
%\end{small}
%\begin{center}
%\includegraphics[width=.44\textwidth,height=.32\textheight,type=png,ext=.png,read=.png]{/afs/cern.ch/user/a/atpathak/afswork/public/Pixel/ALLTXT_22May2017_r9139/InDetAmbiguitySolverSLHC_LinearFit}
%\includegraphics[width=.44\textwidth,height=.32\textheight,type=png,ext=.png,read=.png]{/afs/cern.ch/user/a/atpathak/afswork/public/Pixel/ALLTXT_22May2017_r9139/InDetSCTClusterization_LinearFit} 

%\end{center}
%\item \\
%\end{itemize}
\vspace*{-0.6cm}
\begin{itemize}
\begin{small}
\item This plot shows that the linear distribution between Number of events
  and Fitted mean/event for InDetSiSpTrackFinderSLHC 
\item The probability for the p0 fit is less than the p1 fit that
  indicates p0 is the best fit as compared to p1 
\item Since p0 is the best fit, we can conclude that Total Time taken
  per event is independent of the number of events.  
%\item The following 3 different configurations are studied:
\end{small}
\end{itemize}
\end{normalsize}
%\end{center}
\end{frame}
%------------------------------------------------
\begin{frame}
\frametitle{Fitted mean per event vs Number of events in BCL(r9139) for InDetAmbiguitySolverSLHC}

%\begin{center}
\begin{normalsize}

%\begin{itemize}

%\item Time (minute) / event 
%For AthAlgSeq (left)  and InDetSiSpTrackFinderSLHC (right) \\
%\item Timing for r9060 [left], r9133 [center] and r9139 [right]\\
%\item \\
%\item \\
\vspace*{-0.6cm}
%\begin{center}
%\item For r9060(left) and For r9133(right) \\
%\includegraphics[width=.55\textwidth,height=.42\textheight,type=png,ext=.png,read=.png]{/afs/cern.ch/user/a/atpathak/afswork/public/Pixel/ALLTXT_22May2017_r9139/AthAlgSeq_LinearFit} 
%\includegraphics[width=.44\textwidth,height=.32\textheight,type=png,ext=.png,read=.png]{/afs/cern.ch/user/a/atpathak/afswork/public/Pixel/ALLTXT_22May2017_r9139/InDetSiSpTrackFinderSLHC_LinearFit}
%\end{center}
%\item Time (minute) / event 
%\begin{small} 
%for InDetAmbiguitySolverSLHC(left) and InDetSCT\_Clusterization(right)\\
%\end{small}
\begin{center}
\includegraphics[width=.55\textwidth,height=.42\textheight,type=png,ext=.png,read=.png]{/afs/cern.ch/user/a/atpathak/afswork/public/Pixel/ALLTXT_22May2017_r9139/InDetAmbiguitySolverSLHC_LinearFit}
%\includegraphics[width=.44\textwidth,height=.32\textheight,type=png,ext=.png,read=.png]{/afs/cern.ch/user/a/atpathak/afswork/public/Pixel/ALLTXT_22May2017_r9139/InDetSCTClusterization_LinearFit} 

\end{center}
%\item \\
%\end{itemize}
\vspace*{-0.6cm}
\begin{itemize}
\begin{small}
\item This plot shows that the linear distribution between Number of events
  and Fitted mean/event for InDetAmbiguitySolverSLHC 
\item The probability for the p0 fit is less than the p1 fit that
  indicates p0 is the best fit as compared to p1 
\item Since p0 is the best fit, we can conclude that Total Time taken
  per event is independent of the number of events.  
%\item The following 3 different configurations are studied:
\end{small}
\end{itemize}
\end{normalsize}
%\end{center}
\end{frame}
%------------------------------------------------
\begin{frame}
\frametitle{Fitted mean per event vs Number of events in BCL(r9139) for InDetSCT\_Clusterization}

%\begin{center}
\begin{normalsize}

%\begin{itemize}

%\item Time (minute) / event 
%For AthAlgSeq (left)  and InDetSiSpTrackFinderSLHC (right) \\
%\item Timing for r9060 [left], r9133 [center] and r9139 [right]\\
%\item \\
%\item \\
\vspace*{-0.6cm}
%\begin{center}
%\item For r9060(left) and For r9133(right) \\
%\includegraphics[width=.55\textwidth,height=.42\textheight,type=png,ext=.png,read=.png]{/afs/cern.ch/user/a/atpathak/afswork/public/Pixel/ALLTXT_22May2017_r9139/AthAlgSeq_LinearFit} 
%\includegraphics[width=.44\textwidth,height=.32\textheight,type=png,ext=.png,read=.png]{/afs/cern.ch/user/a/atpathak/afswork/public/Pixel/ALLTXT_22May2017_r9139/InDetSiSpTrackFinderSLHC_LinearFit}
%\end{center}
%\item Time (minute) / event 
%\begin{small} 
%for InDetAmbiguitySolverSLHC(left) and InDetSCT\_Clusterization(right)\\
%\end{small}
\begin{center}
%\includegraphics[width=.44\textwidth,height=.32\textheight,type=png,ext=.png,read=.png]{/afs/cern.ch/user/a/atpathak/afswork/public/Pixel/ALLTXT_22May2017_r9139/InDetAmbiguitySolverSLHC_LinearFit}
\includegraphics[width=.55\textwidth,height=.42\textheight,type=png,ext=.png,read=.png]{/afs/cern.ch/user/a/atpathak/afswork/public/Pixel/ALLTXT_22May2017_r9139/InDetSCTClusterization_LinearFit} 

\end{center}
%\item \\
%\end{itemize}
\vspace*{-0.6cm}
\begin{itemize}
\begin{small}
\item This plot shows that the linear distribution between Number of events
  and Fitted mean/event for InDetSCT\_Clusterization 
\item The probability for the p0 fit is less than the p1 fit that
  indicates p0 is the best fit as compared to p1 
\item Since p0 is the best fit, we can conclude that Total Time taken
  per event is independent of the number of events.  
%\item The following 3 different configurations are studied:
\end{small}
\end{itemize}
\end{normalsize}
%\end{center}
\end{frame}
%------------------------------------------------
\begin{frame}
\frametitle{\begin{normalsize} Summary \end{normalsize}}

\begin{center}
\begin{normalsize}

\begin{itemize}

\item Work in progress to benchmark reconstruction times\\

\item Looked at four different Algorithm for 30 sets jobs for 7 sets
  of events \\

\item CpuEfficiency increases as we increase the number of events

\item Total Time taken per event does not depend on the number of events. 

\item Preparing to produce the plots of the cpu time taken at different mu-values\\

\end{itemize}
\end{normalsize}
\end{center}
\end{frame}
%------------------------------------------------
\end{document}
%------------------------------------------------
\begin{frame}
\frametitle{Introduction}
\vspace*{1cm}
\begin{itemize}
%\item This work is part of my authorship qualification task, which is
  %work in progress
\item Reconstruction times should prefereably be benchmarked in
  a controlled environment, eg. uniform cpuEfficiency/job
\item Many thanks to Dr.~Dang for installing ATLAS software using {\tt
    cvmfs} in the new computing clusters at UofL.
\item This study compares RAWtoESD output run over RDO files 
\item The following 3 different configurations are studied:
\end{itemize}
%\vspace*{.5cm}
\begin{table}
%\caption{Configuration}
%\label{tab:config}
\begin{center}
\begin{tabular}{llll}\hline\hline
Description & Reco release & Reco flag & Input RDO  \\\hline
Nominal     & 20.20.7.4    & AMI=r9060 & r-tag=8832 \\
Long strips & 20.20.7.6    & AMI=r9133 & r-tag=9133 \\
BCL         & 20.20.7.8    & AMI=r9139 & r-tag=9139 \\
\hline \hline
\end{tabular}
\end{center}
\end{table}

\vspace*{-.25cm}
\begin{center}
All 3 configurations use for seeding:
SiCombinatorialTrackFinderTool\_xk-01-00-34-06,
SiSpacePointsSeedTool\_xk-01-00-35-05, 
SiSPSeededTrackFinder-01-00-07
\end{center}

\end{frame}
%------------------------------------------------
\begin{frame}
\frametitle{Reconstruction times for AthAlgSeq}

\begin{center}
\begin{normalsize}

\begin{itemize}

\item Time (minute) / event for 20 sets of jobs, each over 50 events \\
\item Timing for r9060 [left], r9133 [center] and r9139 [right]\\
%\item \\
%\item \\
\vspace*{-.5cm}
\begin{center}
%\item For r9060(left) and For r9133(right) \\
\includegraphics[width=.32\textwidth,height=.4\textheight,type=png,ext=.png,read=.png]{/afs/cern.ch/user/a/atpathak/afswork/public/Pixel/Plots_11May_2016/pic/mydata_AthAlgSeq_r9060} 
\includegraphics[width=.32\textwidth,height=.4\textheight,type=png,ext=.png,read=.png]{/afs/cern.ch/user/a/atpathak/afswork/public/Pixel/Plots_11May_2016/pic/mydata_AthAlgSeq_r9133}
 \includegraphics[width=.32\textwidth,height=.4\textheight,type=png,ext=.png,read=.png]{/afs/cern.ch/user/a/atpathak/afswork/public/Pixel/Plots_11May_2016/pic/mydata_AthAlgSeq_r9139} 
\end{center}
%\item \\
\end{itemize}
\end{normalsize}

\end{center}
%\vspace*{1cm}
\begin{table}
%\caption{Time Taken}
%\label{tab:Time}
\begin{center}
\begin{tabular}{lccc}\hline\hline
Algorithm                        & r9060 & r9133 & r9139 \\\hline
AthAlgSeq:Execute                &  6.185  & 6.522   & 6.41   \\\hline \hline
\end{tabular}
\end{center}
\end{table}
 

\end{frame}
%------------------------------------------------
\begin{frame}
\frametitle{Reconstruction times for InDetSiSpTrackFinderSLHC}

\begin{center}
\begin{normalsize}

\begin{itemize}

\item Time (minute) / event for 20 sets of jobs, each over 50 events \\
\item Timing for r9060 [left], r9133 [center] and r9139 [right]\\
%\item  \\
%\item \\
%\item \\
\vspace*{-.5cm}
\begin{center}
\includegraphics[width=.32\textwidth,height=.4\textheight,type=png,ext=.png,read=.png]{/afs/cern.ch/user/a/atpathak/afswork/public/Pixel/Plots_11May_2016/pic/mydata_InDetSiSpTrackFinderSLHC_r9060} 
\includegraphics[width=.32\textwidth,height=.4\textheight,type=png,ext=.png,read=.png]{/afs/cern.ch/user/a/atpathak/afswork/public/Pixel/Plots_11May_2016/pic/mydata_InDetSiSpTrackFinderSLHC_r9133}
\includegraphics[width=.32\textwidth,height=.4\textheight,type=png,ext=.png,read=.png]{/afs/cern.ch/user/a/atpathak/afswork/public/Pixel/Plots_11May_2016/pic/mydata_InDetSiSpTrackFinderSLHC_r9139} 
\end{center}
%\item \\
\end{itemize}
\end{normalsize}

\end{center}
%\vspace*{1cm}
\begin{table}
%\caption{Time Taken}
%\label{tab:Time}
\begin{center}
\begin{tabular}{lccc}\hline\hline
Algorithm                        & r9060 & r9133 & r9139 \\\hline
InDetSiSpTrackFinderSLHC:Execute & 1.127  & 1.377  & 1.155  \\\hline \hline 
\end{tabular}
\end{center}
\end{table}

\end{frame}
%------------------------------------------------
\begin{frame}
\frametitle{Reconstruction times for InDetAmbiguitySolverSLHC}

\begin{center}
\begin{normalsize}

\begin{itemize}

\item Time (minute) / event for 20 sets of jobs, each over 50 events \\
\item Timing for r9060 [left], r9133 [center] and r9139 [right]\\
%\item  \\
%\item \\
%\item \\
\vspace*{-.5cm}
\begin{center}
\includegraphics[width=.32\textwidth,height=.4\textheight,type=png,ext=.png,read=.png]{/afs/cern.ch/user/a/atpathak/afswork/public/Pixel/Plots_11May_2016/pic/mydata_InDetAmbiguitySolverSLHC_r9060} 
\includegraphics[width=.32\textwidth,height=.4\textheight,type=png,ext=.png,read=.png]{/afs/cern.ch/user/a/atpathak/afswork/public/Pixel/Plots_11May_2016/pic/mydata_InDetAmbiguitySolverSLHC_r9133}
\includegraphics[width=.32\textwidth,height=.4\textheight,type=png,ext=.png,read=.png]{/afs/cern.ch/user/a/atpathak/afswork/public/Pixel/Plots_11May_2016/pic/mydata_InDetAmbiguitySolverSLHC_r9139} 
\end{center}
%\item \\
\end{itemize}
\end{normalsize}

\end{center}

%\vspace*{1cm}
\begin{table}
%\caption{Time Taken}
%\label{tab:Time}
\begin{center}
\begin{tabular}{lccc}\hline\hline
Algorithm                        & r9060 & r9133 & r9139 \\\hline
InDetAmbiguitySolverSLHC:Execute & 0.325  & 0.3375  & 0.335   \\\hline \hline
\end{tabular}
\end{center}
\end{table}

\end{frame}
%------------------------------------------------
\begin{frame}
\frametitle{Reconstruction times for InDetSCT\_Clusterization}

\begin{center}
\begin{normalsize}

\begin{itemize}

\item Time (second) / event for 20 sets of jobs, each over 50 events \\
\item Timing for r9060 [left], r9133 [center] and r9139 [right]\\
%\item  \\
%\item \\
%\item \\
\vspace*{-.5cm}
\begin{center}
\includegraphics[width=.32\textwidth,height=.4\textheight,type=png,ext=.png,read=.png]{/afs/cern.ch/user/a/atpathak/afswork/public/Pixel/Plots_11May_2016/pic/mydata_InDetSCTClusterization_r9060} 
\includegraphics[width=.32\textwidth,height=.4\textheight,type=png,ext=.png,read=.png]{/afs/cern.ch/user/a/atpathak/afswork/public/Pixel/Plots_11May_2016/pic/mydata_InDetSCTClusterization_r9133}
\includegraphics[width=.32\textwidth,height=.4\textheight,type=png,ext=.png,read=.png]{/afs/cern.ch/user/a/atpathak/afswork/public/Pixel/Plots_11May_2016/pic/mydata_InDetSCTClusterization_r9139} 
\end{center}
%\item \\
\end{itemize}
\end{normalsize}

\end{center}

%\vspace*{1cm}
\begin{table}
%\caption{Time Taken}
%\label{tab:Time}
\begin{center}
\begin{tabular}{lccc}\hline\hline
Algorithm                        & r9060 & r9133 & r9139 \\\hline
InDetSCT\_Clusterization:Execute & 0.7933 & 0.8413 & 0.8592 \\\hline \hline
\end{tabular}
\end{center}
\end{table}

\end{frame}
%------------------------------------------------
\begin{frame}
\frametitle{\begin{normalsize} Summary \end{normalsize}}

\vspace*{1cm}
\begin{itemize} 
\item Time per event from study over 1000 events : 20 jobs of 50 events each \\
\end{itemize}
\vspace*{.25cm}
\begin{table}
%\caption{Time Taken}
%\label{tab:Time}
\begin{center}
\begin{tabular}{lccc}\hline\hline
Algorithm                        & r9060 & r9133 & r9139 \\\hline
AthAlgSeq:Execute(min)  &  6.185  & 6.522   & 6.41   \\ 
InDetSiSpTrackFinderSLHC:Execute(min) & 1.127  & 1.377  & 1.155  \\ 
InDetAmbiguitySolverSLHC:Execute(min) & 0.325  & 0.3375  & 0.335  \\
InDetSCT\_Clusterization:Execute(sec) & 0.7933 & 0.8413 & 0.8592 \\\hline \hline
\end{tabular}
\end{center}
\end{table}

\vspace*{1cm}

\end{frame}
%------------------------------------------------
\end{document}
%------------------------------------------------
\begin{frame}
\frametitle{\begin{normalsize} Summary \end{normalsize}}

\begin{center}
\begin{normalsize}

\begin{itemize}

\item Work in progress to benchmark reconstruction times\\

\item Looked at four different Algorithm for 20 sets of 50 events jobs \\

%\item Cannot yet make firm conclusion in the trend observed

%\item Large variations are seen in 2 sets of jobs [50 events each]\\

%\item Preparing to run 20 sets of jobs [50 events each] \& average\\

\end{itemize}
\end{normalsize}
\end{center}
\end{frame}
%------------------------------------------------
\end{document}
%------------------------------------------------
\begin{frame}
\frametitle{Reconstruction times}

\vspace*{1cm}  
Average time (seconds) per event for job \# 2 run over 50 events

\vspace*{1cm}  
\begin{table}
%\caption{Time Taken}
%\label{tab:Time}
\begin{center}
\begin{tabular}{lccc}\hline\hline
Algorithm                        & r9060 & r9133 & r9139 \\\hline
AthAlgSeq:Execute                &  365  & 340   & 366   \\ 
InDetSiSpTrackFinderSLHC:Execute & 68.5  & 75.2  & 67.0  \\ 
InDetAmbiguitySolverSLHC:Execute & 18.8  & 18.1  & 19.1  \\
InDetSCT\_Clusterization:Execute & 0.789 & 0.757 & 0.794 \\ \hline \hline
\end{tabular}
\end{center}
\end{table}

\vspace*{1cm}
\end{frame}
%------------------------------------------------
\begin{frame}
\frametitle{Reconstruction times}

\vspace*{1cm} 
Average time (seconds) per event averaged over 100 events

\vspace*{1cm}
\begin{table}
%\caption{Time Taken}
%\label{tab:Time}
\begin{center}
\begin{tabular}{lccc}\hline\hline
Algorithm                        & r9060 & r9133 & r9139 \\\hline
AthAlgSeq:Execute                &  381  & 365   & 383.5 \\ 
InDetSiSpTrackFinderSLHC:Execute & 69.6  & 79.1  & 67.8  \\ 
InDetAmbiguitySolverSLHC:Execute & 19.8  & 18.6  & 19.0  \\
InDetSCT\_Clusterization:Execute & 0.874 & 0.767 & 0.794 \\\hline \hline
\end{tabular}
\end{center}
\end{table}

\vspace*{1cm}
\end{frame}
%------------------------------------------------
\begin{frame}
\frametitle{\begin{normalsize} Summary \end{normalsize}}

\begin{center}
\begin{normalsize}

\begin{itemize}

\item Work in progress to benchmark reconstruction times\\

\item Cannot yet make firm conclusion in the trend observed

\item Large variations are seen in 2 sets of jobs [50 events each]\\

\item Preparing to run 20 sets of jobs [50 events each] \& average\\

\end{itemize}
\end{normalsize}
\end{center}
\end{frame}
%------------------------------------------------
\end{document}
%------------------------------